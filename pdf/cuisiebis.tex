\documentclass{C:/Users/Simon/Desktop/mamette/cuisine/recipe}
\usepackage{nicefrac} %Pour les fractions
\usepackage[utf8]{inputenc} % Pour avoir
\usepackage[francais]{babel} % les accents français
\usepackage{graphicx} % Pour inclure l'image à la fin
\usepackage{lipsum}
\usepackage{charter}
\usepackage{pdfpages}

\usepackage{etoolbox}
\makeatletter
\patchcmd{\chapter}{\if@openright\cleardoublepage\else\clearpage\fi}{}{}{}
\makeatother


\usepackage{titlesec, blindtext, color}
\definecolor{gray75}{gray}{0.75}
\newcommand{\hsp}{\hspace{20pt}}
\titleformat{\chapter}[hang]{\Huge\bfseries}{\thechapter\hsp\textcolor{gray75}{|}\hsp}{0pt}{\Huge\bfseries}

\usepackage{tocloft}

\setlength\cftaftertoctitleskip{0pt}

\setcounter{secnumdepth}{0}


\DeclareUnicodeCharacter{00A0}{ }

\setlength{\parindent}{0pt}

\begin{document}
\frontmatter
\pagestyle{empty}
\begin{titlepage}
\includepdf{images/couverture2.pdf}
\end{titlepage}

\begin{titlepage}
\parindent=0pt
 \hspace*{\stretch{1}}  %
 \hspace*{\stretch{1}} 
\vspace*{\stretch{1}}

\begin{center}\bfseries\Huge
    Le petit carnet de recettes\\de Mamette
\end{center}
\hrulefill
\vspace*{1cm}
\begin{center}\bfseries\Large
Noël 2015
\end{center}
    
\begin{center}\bfseries\Large
Alice \textsc{Picard}\\
Hanna \textsc{Picard}\\
Mila \textsc{Picard}\\
Nissim \textsc{Picard}\\
Simon \textsc{Picard}
\end{center}
    \vspace*{\stretch{2}}
\begin{flushright}
       
\end{flushright}   
\end{titlepage}
\cleardoublepage

\vspace*{\fill}
\begin{center}
Les prémices de cette retranscription sont nées au Japon lors ce que Alice et Simon étaient dans le monorail de Tokyo, survolant Odaiba. Ce projet est né de l'envie commune de pouvoir profiter des trésors culinaires figurant dans le carnet de leur grand-mère, même s'il est évident qu'ils ne pourront qu'approcher le savoir-faire de Mamette.\\
Ces deux derniers ont rapidement contacté leurs cousins pour mener à bien ce plan, ceux-ci ayant accepté avec enthousiasme.\\


C'est donc avec plaisir et dans une complicité familiale que nous vous proposons de découvrir ce qui se cache effectivement dans ce fameux carnet, en espérant que, comme nous, vous pourrez profiter de cette cuisine légendaire.
\end{center}
\vspace*{\fill}
\cleardoublepage


\tableofcontents
\clearpage

\mainmatter% corps du document
\pagestyle{headings}

\chapter{Entrées}
\begin{figure}[h]
\centering
\includegraphics[width=\textwidth]{images/chapter1.jpg}
\end{figure}
\begin{minipage}[c]{\textwidth}
\recette{Potage au curry}
Couper en petits dés : 4 carottes, 2 blancs de poireaux. Emincer un oignon et une gousse d’ail. Cuire $\nicefrac{1}{2}$ heure à petit feu dans un fond d’eau avec 2 cubes de bouillon. Ajouter une cuillère à dessert de curry. \\
Passer au mixage de la soupe. \\
\\

\end{minipage}

\begin{minipage}[c]{\textwidth}
\recette{Mousse de saumon}
\ingredients{
\item Une boite de saumon au naturel 425-450 gr
\item Un sachet de gelée royale
\item Une petite boite de crème épaisse
\item Citron
\item Poivre
}
Diluer la gelée dans $\nicefrac{1}{2}$ L d’eau. Porter à ébullition puis retirer du feu. \\
Ajouter la crème, le saumon égoutté, $\nicefrac{1}{2}$ jus de citron et le poivre. Bien mixer le tout. \\
Mettre au frigo au moins 5h. \\
\\

\end{minipage}

\begin{minipage}[c]{\textwidth}
\recette{Mousse aux foies de volaille}
\ingredients{
\item +- 450 gr de foie
\item 1 sachet de gelée royale
\item 1 petite boite de crème épaisse
\item cognac ou porto
}
Rôtir le foie jusqu’à ce qu’il soit bien cuit. \\
Flamber au cognac ou mettre du porto, ajouter du sel et du poivre. \\
Continuer comme la recette précédente, sans le citron bien sûr. \\
\\

\end{minipage}

\begin{minipage}[c]{\textwidth}
\recette{Salade italienne (excellent !) }
\otor{Bonne Maman Picard}
Couper en petits morceaux : \\
6 grosses pommes de terre cuites\\
La même quantité de betteraves rouges cuites (ou un pot de betteraves au vinaigre)\\
1 dés de céleri rave cuit\\
1 oignon\\
1 pomme crue\\
$\nicefrac{1}{2}$ concombre\\
1 à 2 cuillères à soupe de petits pois (boite)\\
1 boite d’anchois à l’huile\\
Mélanger avec poivre et sel\\
1 à 2 cuillères à soupe de viande cuite\\
Mélanger avec mayonnaise épaisse\\
Décorer avec œuf dur haché, persil, betterave rouge hachée\\
\\

\end{minipage}

\begin{minipage}[c]{\textwidth}
\recette{Soupe à l'orge}
Cuire pendant +- 1 heure de l'orge dans de l'eau avec une gousse d'ail entière, du thym et du laurier. Ne pas saler.\\
Cuire 20 minutes dans de l'eau quelques pommes de terre et un cèleri coupé en morceaux. Saler.\\
Mélanger les deux.\\
\\

\end{minipage}

\begin{minipage}[c]{\textwidth}
\recette{Consommé fin }
\otor{Tilou}
Après les 2h30 de cuisson, ajouter 1 livre ? de bœuf haché et laisser le tout encore cuire 1h.\\
Enlever la viande avant de servir et l’utiliser comme les restes de viande.\\
\\

\end{minipage}

\begin{minipage}[c]{\textwidth}
\recette{Soupe à l’oignon (gruau) }
\otor{Tilou}
Faire revenir un gros oignon dans du beurre (non-bruni !). Le laisser suer à feu très doux. Couvrir d’eau. Ajouter un céleri, une poignée de gruau, du poivre et du sel.\\
Laisser le tout cuire environ 25 minutes. \\
Passer le potage avant de servir.\\
\\

\end{minipage}

\begin{minipage}[c]{\textwidth}
\recette{Soupe au cresson }
\otor{Tilou}
Jeter dans l’eau une botte de cresson, un céleri coupé, du poivre, du sel, du thym, du laurier et une gousse d’ail.\\
Ajouter une poignée de riz quand ça bout.\\
Après environ 25 minutes de cuisson, passer le tout.\\
Au moment de servir, ajouter du liebig et éventuellement un jaune d’œuf dans le fond de la soupière.\\
\\

\end{minipage}

\begin{minipage}[c]{\textwidth}
\recette{Fondues au fromage}
Même pâte que pour les croquettes aux crevettes mais sans les crevettes et en doublant la quantité de fromage (100 gr de gruyère râpé et 50 gr de parmesan râpé). \\
\\
NB : Toujours mettre à demi moins de parmesan par rapport aux autres fromages.\\
\\

\end{minipage}

\begin{minipage}[c]{\textwidth}
\recette{Bouillon }
\otor{Tilou}
Dans l’eau chaude mettre 2-3 carotte, 2-3 poireaux et 1 céleri (environ le tout par parts égales). \\
1,5 livre ? de bouillis au moins.\\
1 livre ? d’os à moelle si possible.\\
Du poivre, du sel, du thym, du laurier et de l’ail.\\
Faire brunir 1 oignon au four jusqu’à ce qu’il soit bien brun.\\
2h30 de cuisson douce.\\
Porter le tout à ébullition puis sur flamme frisonnante.\\
Ajouter des pâtes au moment de servir.\\
\\

\end{minipage}

\begin{minipage}[c]{\textwidth}
\recette{Croquettes aux crevettes}
100 gr de crevettes épluchées (ce qui correspond environ à 250 gr de crevettes non épluchées).  \\
Les faire suer dans un poêlon couvert à feu très doux. \\
Dans un autre poêlon, faire fondre 100 gr de beurre. Y ajouter 100 gr de farine et les trois quarts d’une pinte de lait chaud, du sel, du poivre et 50 gr de fromage. Bien mélanger le tout jusqu’à ce que la pâte se détache de la casserole (sinon elle casse dans la friture). \\
Hors du feu, incorporer un à un, 2 ou 3 jaunes d’œufs à la pâte puis les crevettes.  \\
Remettre un instant sur le feu. \\
Mettre la pâte sur une plaque huilée et mouler les croquettes à l’épaisseur voulue. \\
Laisser refroidir.\\
Passer les croquettes dans la farine, le blanc d’œuf et la chapelure.\\
Laisser reposer.\\
Cuire à la friture.\\
\\

\end{minipage}

\begin{minipage}[c]{\textwidth}
\recette{Bouchées aux crevettes}
Remplir les ridés de la même pâte que pour les croquettes aux crevettes et passer les au four 5 minutes. \\
\\

\end{minipage}

\begin{minipage}[c]{\textwidth}
\recette{Riz de veau aux champignons}
Etape 1: La viande\\
Mettre dégorger la viande dans de l’eau salée ($\nicefrac{1}{2}$ cuillère à soupe de sel).\\
Changer d’eau. \\
Mettre la viande à cuire avec un oignon, une feuille de thym et de laurier, 2 pincées de sel et de poivre.\\
Cuisson une grosse demi-heure.\\
Décortiquer la viande une fois cuite. \\
\\
Etape 2: La sauce (200 gr. de sauce aux champignons pour 350 gr. de riz de veau)\\
Bien laver les champignons à plusieurs eaux.\\
Couper les pieds des têtes, puis égoutter. Faire fondre un gros morceau de beurre jusqu’à ce que ça devienne mousseux. \\
Y faire sauter les champignons à vive allure jusqu’à absorption de tout le beurre. \\
Ajouter 2 cuillères à soupe de farine et $\nicefrac{1}{2}$ jus de citron. Les incorporer totalement. \\
Bien couvrir de lait. Cuire à vive allure 5 minutes en secouant la casserole de temps en temps puis cuire à feu doux 20 à 30 minutes.\\
\\
Etape 3: Les boulettes\\
Pour allonger on peut mettre des boulettes de veau.\\
Mélanger 100 gr de viande de veau haché, 1 jaune d’œuf, du sel et du poivre. \\
Ajouter une poignée de persil haché. Faire absorber du lait par une tranche de mie de pain     au feu avant de l'incorporer à la viande. \\
Former des petites boulettes de viande.\\
Les cuire 5 minutes dans du bouillon (ou un cube dans de l’eau bouillante).\\
\\
Ajouter le riz de veau en tranches et les boulettes à la sauce aux champis ! Bon appétit \\
\\

\end{minipage}

\begin{minipage}[c]{\textwidth}
\recette{Champignons à la grecque}
Dans une casserole versez :\\
un verre d'eau et demie, un demie verre d'huile, un jus de citron, une gousse d'ail brouillé, 6 grains de poivre, 6 grains de cerfeuil, un brins de fenouil, du thym et du laurier. Mélanger et y faire bouillir des petits champignons ou des champignons coupés.\\
Laisser cuire 8 min.\\
Servir bien frais.\\
\\

\end{minipage}

\begin{minipage}[c]{\textwidth}
\recette{Soupe à l'orge}
Faire cuire de l'orge dans de l'eau avec une gouse d'ail entière, du thym et du laurier. Ne pas saler. Faire cuire pendant 1h environ. Faire cuire 20 min dans de l'eau quelques pommes de terre et un cèleri coupé en morceaux, saler et ensuite les rajouter à la soupe.\\
\\

\end{minipage}

\begin{minipage}[c]{\textwidth}
\recette{Potage au pourpier}
Cuir du riz dans du bouillon, avec un cèleri coupé en morceaux. Passez et assaisonnez.\\
Y ajouter le pourpier avant de servir.\\
\\

\end{minipage}

\begin{minipage}[c]{\textwidth}
\recette{Pourpier en légume}
Nettoyer le pourpier divisé en feuilles, le cuire dans du beurre mousseux ,environ 1 litre pour deux personnes.\\
Cuir environ 10 minutes.\\
\\

\end{minipage}

\begin{minipage}[c]{\textwidth}
\recette{Concombre à la grec}
Hors d'œuvre à servir glacé.\\
Détaillez en quartiers deux concombres blancs. Mettre dans une cuisson bouillante de 4dc d'eau, 1dcl d'huile et deux jus de citron, un bouquet garni composé de cèleri, de fenouil, de persil, de thym et de laurier.\\
Ajouter 20 graines de cerfeuil, du poivre et du sel.\\
Cuire à vive allure 10 à 12 minutes.\\
Retirez le bouquet et laissez bien refroidir.\\
\\

\end{minipage}

\begin{minipage}[c]{\textwidth}
\recette{Prunes au vinaigre}
Piquer 4kg de prunes et les équeuter. Ajouter deux litres de vinaigre de vin, 1 kg de sucre Candy blanc, du poivre, deux clous de girofle, du thym et du macis (écorce de muscadier).\\
Le sirop doit bouillir 4 minutes. Ajouter les fruits et faire bouillir. Débarrasser dans une urne.\\
Deux jour après, rangez les prunes en bocaux. Refaire bouillir le jus que vous versez sur les prunes. Bien boucher solidement le lendemain\\
\\

\end{minipage}

\begin{minipage}[c]{\textwidth}
\recette{Morilles}
Les laver à plusieurs reprises à l’eau avec grand soin. Bien éponger. Les sauter au beurre. Comme elles donnent beaucoup d'eau, les faire sauter deux fois puis les égoutter jusqu'à ce qu'elles nagent.\\
Les jeter dans un nouveau beurre brulant pour finir la cuisson.\\
(l'eau sert pour un potage ou une sauce)\\
\\

\end{minipage}

\begin{minipage}[c]{\textwidth}
\recette{Potage de tomate passée}
Ajouter 2 lanières de poivron finement coupé et un peu de riz.\\
Jeter le tout dans la soupe\\
\\

\end{minipage}

\begin{minipage}[c]{\textwidth}
\recette{Cèpes à la bordelaise}
Faire sauter dans de l'huile d'olive bien chaude les têtes de cèpes pendant 10 minutes.\\
Quand elles sont dorées, ajouter de l'ail, du persil, les queues hachées, du sel, du poivre et un jus de citron. Servir bien chaud\\
\\

\end{minipage}

\begin{minipage}[c]{\textwidth}
\recette{Potage tortue}
Faire colorer dans une cuillère a soupe : du beurre, 1 kg d'os concassé, une carotte coupée en petits morceaux, 1 oignon, 1 poireau, 1 petit cèleri et 1 gousse d'ail.\\
Puis versé 2 L d'eau, du thym, du laurier, sel, poivre.\\
Faire cuire à cuisson lente pendant 2h puis ajouter 1 dcl de purée de tomate. Cuire encore 15 min. Passer finement. Couper en dés 100g de tête de veau cuite avec 1 dcl de madere et deux jaunes d'oeufs durs. Couper en quartiers puis chauffer.\\
Lier le potage avec une cuillère a soupe et demie de fécule, délayer dans du madere.\\
Rebouillir 5 min.\\
\\

\end{minipage}

\begin{minipage}[c]{\textwidth}
\recette{Vrai consommé}
Dans un bouillon (cuit 2h30) ajouter une tranche de culotte de bœuf haché de 500g trituré avec deux blancs d'œufs, 1 blanc de poireau haché, une carotte et remettre à cuire pendant 1h30, pas plus !\\
\\

\end{minipage}

\begin{minipage}[c]{\textwidth}
\recette{Potage au champignons}
Dans le bouillon d'oignon verser les champignons hachés et cuits à l'eau légèrement salée avec leur eau de cuisson ou tout simplement l'eau de cuisson seule et employer les champignon pour autre chose, lié avec de la farine.\\
\\

\end{minipage}

\begin{minipage}[c]{\textwidth}
\recette{Cassolettes de saint Flour }
Hacher grossièrement 125g de fromage cantal. Y mélanger dans une terrine bien énergiquement : 250g de fromage blanc égoutté, 2 jaunes d'oeufs et un blanc,1/2 cuillère à café de sel fin,6 à 7 tours de moulin à poivre,1 cuillère à café de ciboulettes haché.\\
Distribuer le tout dans des ramequins. Cuire à feu vif pendant 20 min. Servir en saupoudrant de deux cuillères à café de ciboulette hachées.\\
NB : peut se faire avec du gruyère, du reblochon ou autre fromage cuit\\
\\

\end{minipage}

\begin{minipage}[c]{\textwidth}
\recette{Potage aux lentilles}
Avec des os de mouton, faite un bouillon avec un oignons, deux blancs de poireaux, du thym et du laurier. Cuire 9 pommes de terres. Cuire 250g de lentilles à l'eau froide.\\
Ajouter le tout au reste pour cuire ensemble.\\
\\

\end{minipage}

\begin{minipage}[c]{\textwidth}
\recette{Potage au gruau}
Faire revenir quelques minutes un petit oignon dans une noix de beurre mousseuse.\\
Couvrir d'eau. Jeter y un cèleri finement coupé et une forte pognée de gruaux, du thym, du laurier de l’ail et un cube.\\
Passer et servez\\
\\

\end{minipage}

\begin{minipage}[c]{\textwidth}
\recette{Potage à l'oignion}
Faire revenir dans du beurre mousseau deux oignons hachés jusqu'a ce qu'ils donnent leur suc. Mouiller d'eau bouillante. Laisser cuire 2 minutes. Egoutter. Lier le jus avec 3 cuillères de farine bien délayée et un morceau de liebig.\\
Faire un bon bouillon et servez avec du fromage râpé.\\
\\

\end{minipage}

\begin{minipage}[c]{\textwidth}
\recette{Potage longchamp}
250g de pois réduis en purée\\
100g d’oseilles ? cuites à part au beurre bien fondu. Passer l'oseille et les pois.\\
D’autre part dans deux cubes Liebig, faire cuire du vermissels.\\
Ajouter au potage passé, une noix de beurre et une cuillère de cerfeuil haché, de la\\
crème ou du lait facultatif.\\
\\

\end{minipage}

\begin{minipage}[c]{\textwidth}
\recette{Coulis niçois}
Faire revenir 100g d'oignon dans du beurre mousseux, saupoudrer d’une cuillère de farine. Ensuite, ajouter une boite de purée de tomates ou 5 tomates fraiches et une tranche d'estragon.\\
Verser par dessus 1/2 litre de bouillon ou cube.\\
Finir par un morceau de beurre et servir.\\
\\

\end{minipage}

\begin{minipage}[c]{\textwidth}
\recette{Potage Longchamp}
\ingredients{
\item 250g de pois réduis en purée
\item 100g d’oseille cuite à part au beurre bien fondu
}
Passer l'oseille et les pois\\
D’autre part dans deux cube Liebig faite cuire des vermicelles\\
Ajouter au potage passé + une noix de beurre et une cuillère de cerfeuil haché\\
Crème ou lait facultatif\\
\\

\end{minipage}

\begin{minipage}[c]{\textwidth}
\recette{Potage aux champignons}
Dans le bouillon d'oignions verser les champignons haché est cuit à l'eau légèrement salé\\
Avec leur eau de cuisson ou tout simplement l'eau de cuisson seul et employer les champignons pour autre chose, lié avec de la farine\\
\\

\end{minipage}

\begin{minipage}[c]{\textwidth}
\recette{Potage à l'oignions}
Faire revenir dans du beurre mousseau deux oignions hachée jusqu'à ce qu'ils donnent leur suc\\
Mouiller d'eau bouillante\\
Laisser cuire 2 minutes\\
Égoutter\\
Lier le jus avec 3 cuillères de farine bien délayé\\
Un morceau de Liebig\\
Faite un bon bouillon et servez avec du fromage râpé\\
\\

\end{minipage}

\begin{minipage}[c]{\textwidth}
\recette{Coulis niçois}
Fais revenir 100g d'oignions dans du beurre mousseux\\
Saupoudrez une cuillère de farine\\
Ensuite mettez une boite de purée de tomate ou 5 tomates fraîches et une tranche d'estragon\\
Verser par dessus 1/2 litre de bouillon ou cube\\
Finir par un morceau de beurre et servir\\
\\

\end{minipage}

\begin{minipage}[c]{\textwidth}
\recette{Œufs Mirabeau}
Longuement battre 4 œufs avec un demi verre de lait tiède\\
150g de fromage râpé\\
Une cuillère de beurre fondu et du poivre\\
Mettre dans des ramequins beurrés\\
Cuire au bain mari dans 2 cm d'eau au four 15 à 20 minutes\\
démouler, servir avec de la béchamel ou de la sauce tomate\\
\\

\end{minipage}

\begin{minipage}[c]{\textwidth}
\recette{Œufs mollet cressonnières}
Cuire les œufs 5 min et les rafraichir avant de les pelés\\
Nettoyer une botte de cressons\\
La jeter dans de l'eau bouillante salée\\
Égoutter de suite et rafraichir\\
Exprimé? L’eau \\
Passer au tournis et mélanger à une sauce mayonnaise très relevé\\
\\

\end{minipage}

\begin{minipage}[c]{\textwidth}
\recette{Guacamole (entré mexicaine)}
Ecraser 3 avocats mur\\
Mélanger avec 25g d'oignons rouge hachée,\\
Quelques gouttes de tabasco,\\
Le jus d’ un demie citron\\
1/2 de cuillère à café d'ail en poudre,\\
1/2 cuillère à café de curry en poudre\\
Une pincée de sel\\
\\

\end{minipage}

\begin{minipage}[c]{\textwidth}
\recette{Confiture d'oignions}
Émincé 2 gros oignions ou plus\\
Les faire revenir dans une noix de beurre en les saupoudrant de 2 cuillère à soupe de sucre\\
Laisser cuire 10 min\\
Les mouiller avec 4 cuillère à soupe de vinaigre de vin, un demie litre de vin rouge, 4 cuillère à soupe de grenadine ou 4 cuillère à café de gelé de groseille.\\
Saupoudrez de thym laurier sel poivre\\
Laisser réduire 1/2 h jusqu'à obtenir une masse rouge\\
\\

\end{minipage}

\begin{minipage}[c]{\textwidth}
\recette{Beignet soufflé au parmesan}
Mettre dans une casserole 60g de beurre, du sel, du poivre, de la muscat, du paprika et les 3/4 d'un verre contenant un mélange lait et eau\\
Faire bouillir\\
Ajouter en remuant a la spatule 150g de farine jusqu’à obtenir une pâte épaisse\\
Continuer à battre jusqu’à ce que la pâte épaisse se détache en boule de la casserole\\
Hors du feu ajouter deux œufs à la fois en remuant jusqu'à   ce qu'elle redevienne épaisse. Ajouter en battant 60g de parmesan râpé\\
Cuire à la friture des boulettes de pate grosse comme une noix\\
Servez bien chaud sur serviette pliée ou papier absorbant\\
\\

\end{minipage}

\begin{minipage}[c]{\textwidth}
\recette{Conserve de cornichons et petits oignons}
Les petits oignons doivent être pelés. Les mettre dans une terrine avec une poignée de sel - bien mélanger - laisser reposer une nuit. \\
Le lendemain, bien essuyer 1 ou deux fois selon la saleté. Les mettre en bocal. Ajouter du vinaigre, de l'estragon, du fenouil  éventuellement un piment.\\
\\

\end{minipage}

\begin{minipage}[c]{\textwidth}
\recette{Brochette créole }
\ingredients[(4 personnes)]{
\item 16 petits cubes de gouda d’environ 1 cm 50
\item 16 très fine tranches de lard fumé
\item 4 tranches d'ananas et 4 piques à brochette
}
Enroulez les cubes de fromage entièrement dans le lard\\
Coupé les tranche d'ananas En cube,\\
Enfilé alternativement les cube de fromage enroulé de lard et ceux d'ananas\\
Mettre 10 minutes au grill ou sur un poêle chaude\\
Retourner à mi-cuisson\\
Servir très chaud (avec du riz ça peut faire un repas complet)\\
\\

\end{minipage}



\chapter{Plats}
\begin{figure}[h]
\centering
\includegraphics[width=0.85\textwidth]{images/chapter2.jpg}
\end{figure}
\include{tex/plat}

\chapter{Sauces}
\begin{figure}[h]
\centering
\includegraphics[width=\textwidth]{images/chapter3.jpg}
\end{figure}
\begin{minipage}[c]{\textwidth}
\recette{Sauce vinaigrette}
\ingredients{
\item Huile. 
\item 1 cuillère à café de moutarde.
\item 1 bonne cuillère à soupe de vinaigre.
}
Mélanger le tout en battant énergiquement pour obtenir plus ou moins la même consistance qu’une mayonnaise.\\
\\

\end{minipage}

\begin{minipage}[c]{\textwidth}
\recette{Sauce béarnaise}
\ingredients{
\item 2 jaunes d’œufs
\item 2 cuillères à soupe de sauce (quelle sauce ?) 
\item 1 pincée de sel. 
}
Mélanger le tout à l’aide d’un fouet. Cuire à feu doux en tournant toujours (ne pas laisser bouillir !). Ajouter du beurre par petits morceaux tout en fouettant.\\
\\

\end{minipage}

\begin{minipage}[c]{\textwidth}
\recette{Sauce blanche pour légumes}
Fondre du beurre.\\
Y ajouter $\nicefrac{1}{2}$ cuillère à soupe de farine et une pincée de sel.\\
Sur feu doux ajouter du lait en tournant. \\
Y Verser les légumes cuits à l’eau.\\
\\
NB : pour les épinards ajouter 1 cuillère à soupe de farine en plus pour allonger le légume. \\
\\

\end{minipage}

\begin{minipage}[c]{\textwidth}
\recette{Sauce mousseline (artichauts)}
Faire fondre au bain-marie 150 gr de beurre.\\
Faire réduire de moitié 4 cuillères à soupe de vinaigre.\\
Laisser tiédir.\\
Ajouter 2 jaunes d’œufs en fouettant.\\
Ajouter peu à peu le beurre fondu en fouettant toujours. \\
Saler.\\
Ajouter un peu de jus de citron.\\
Ajouter le reste du beurre.\\
Garder au chaud au bain-marie.\\
\\

\end{minipage}

\begin{minipage}[c]{\textwidth}
\recette{Sauce hollandaise (asperges) }
Mettre dans un poêlon $\nicefrac{1}{2}$ cuillère à soupe d’eau, 1 forte pincée de sel, le jus d’$\nicefrac{1}{2}$ citron (environ 1 cuillère à soupe) et 1 jaune d’œuf. \\
Mélanger alors seulement au fouet.\\
Mettre à feu doux en mélangeant au fouet jusqu’à début d’épaississement.\\
Incorporer peu à peu 60 gr de beurre en mélangeant bien, à très petit feu. \\
Mélanger sans arrêt.\\
\\
NB : Ne pas verser  la sauce dans une saucière plus chaude que la sauce.\\
\\

\end{minipage}

\begin{minipage}[c]{\textwidth}
\recette{Fines herbes}
Mélanger un peu de cressonnette, du persil, du cerfeuil, quelques feuilles d'épinard et d’oseille, 1 ou 2 petits oignons et 1 ou 2 feuilles d’artichaut (sans la partie blanche). Le tout doit peser environ 25 gr.\\
Hacher fin.  \\
\\
Peut s'employer par exemple en mélangeant les fines herbes à de la mayonnaise et en versant le mélange sur de la tête de veau (environ 150 gr par personnes) coupée en cube.\\
\\

\end{minipage}

\begin{minipage}[c]{\textwidth}
\recette{Sauce tartare}
\ingredients{
\item $\nicefrac{1}{4}$ L de mayonnaise
\item 1 gros oignon haché fin
\item 15 gr de feuilles d’estragon frais au vinaigre hachées finement
\item 1 cuillère à dessert de sauce anglaise
\item 25 gr de cornichons hachés 
\item 1 cuillère à dessert câpres hachées
\item Poivre et sel
\item $\nicefrac{1}{2}$ cuillère à café de vinaigre à l’estragon 
\item 1 cuillère à café de moutarde
}
Mélanger au fouet et servir bien frais ! \\
\\

\end{minipage}

\begin{minipage}[c]{\textwidth}
\recette{Ketchup aux pommes}
Prendre 12 pommes pelées, vidées et découpées en quartier. Les couvrir d'eau et laisser mijoter jusqu'a ce qu'elles deviennent molles, passer le tout, ajouter par litre de pomme :\\
une tasse de sucre\\
une cuillère à thé de moutarde\\
une cuillère à soupe de sel\\
une cuillère à thé de clous de girofles\\
2 cuillère à thé de canelle\\
2 tasses de vinaigre de cidre\\
2 oignions râpés\\
Laisser mijoter pendant une heure. Mettre en bouteille.\\
\\

\end{minipage}




\chapter{Désserts}
\begin{figure}[h]
\centering
\includegraphics[width=\textwidth]{images/chapter4.jpg}
\end{figure}
\begin{minipage}[c]{\textwidth}
\recette{Puits d’amour }
\ingredients[(par personne)]{
\item 1 blanc d’œuf en neige + 1 cuillère à soupe de sucre. 
\item 1 jaune + 1 cuillère à soupe de sucre. 
}
Mélanger au fouet jusqu’à ce que ca cloque. Verser le jaune dans le fond du bol. \\
Ajouter une cuillère à café de cognac ou de rhum et/ou une cuillère à café de gelée de groseille. \\
Ajouter le blanc en neige. \\
Un peu de gelée de groseille au sommet. \\
Chacun mélange dans son bol à table !\\
\\

\end{minipage}

\begin{minipage}[c]{\textwidth}
\recette{Äpfel auf der Schuëssel}
Peler les pommes et couper la pomme vidée en 2 krizt ?\\
Beurrer le moule et le mettre au four. Y mettre les pommes + un morceau de beurre, et « un rien » d’eau. \\
Sucrer. \\
Cuire +- 10 minutes. \\
Retirer du four et combler les trous avec de la confiture. \\
Battre 2-3 jaunes d’œufs avec du sucre et de la farine (très peu). Ajouter un zeste de citron. \\
En faire une crème.  \\
Battre les blancs en neige. \\
Mélanger, puis verser le tout sur les pommes. \\
Remettre à feu doux 10 à 15 minutes. \\
\\

\end{minipage}

\begin{minipage}[c]{\textwidth}
\recette{Crème d’amande }
Piller 100g d’amandes avec 100g de sucre. Ajouter 40g de beurre ramolli, deux jaunes d’œufs et deux cuillères de Rhum. Bien mélanger à la spatule. Utiliser sans cuisson.\\
\\

\end{minipage}

\begin{minipage}[c]{\textwidth}
\recette{Purée de châtaignes}
Laver les châtaignes. Les fendre tout autour pour couper les 2 peaux. \\
Jeter dans l’eau bouillante et faites repartir très rapidement l’ébullition. \\
Quand les deux peaux sont ouvertes (3 à 4 minutes), les égoutter, et couvrir pour que l’eau ne s’évapore pas et finir d’éplucher. \\
Jeter les châtaignes épluchées dans l’eau bouillante. Faire repartir l’ébullition mais la régler lentement. \\
Après 15 à 20 minutes d’ébullition (quand une dent de fourchette peut y pénétrer), égoutter au max. \\
Ne pas cuire trop longtemps. Si il y a un peu trop d’eau, ca empêche d’avoir une purée sèche…\\
Ecraser au moulin-légumes 3 fois de suite avec grilles de plus en plus fine : ca donne de la purée. \\
! La purée à manger telle quelle peut être faite au lait !\\
\\

\end{minipage}

\begin{minipage}[c]{\textwidth}
\recette{Petites poires en châtaignes}
\ingredients{
\item 1 kg  de purée sèche de châtaignes (recette précédente) 
\item 100 gr de sucre
\item 30/40 kg de beurre
\item 100 gr de choco râpé 
\item 1 jaune d’œuf
\item 50 gr d’angélique 
}
Mélanger la purée, le sucre, le jaune d œuf et le beurre. \\
Faire de ce mélange des boulettes très sucrées qu’on moule en forme de poires. \\
Rouler celles-ci dans le chocó rapé. \\
Piquer le bâtonnet d’angélique en queue.\\
\\

\end{minipage}

\begin{minipage}[c]{\textwidth}
\recette{Mousse de pommes }
\ingredients[(6 à 8 personnes)]{
\item Pommes reinettes : 1,2 kg
\item Sucre en poudre : 150 gr
\item Gélatine : 7 feuilles
\item Eau chaude : 1 cuillère a soupe
\item Pour le caramel : sucre (100-150 gr) et une cuillère à soupe d’eau.
}
Eplucher les pommes, retirer les pépins. Faire de la compote et y ajouter 150 gr de sucre. \\
Cuire 10 minutes. \\
Tremper la gélatine 10 à 15 minutes dans de l’eau froide. \\
Ajouter dans la casserole à une cuillère à soupe d’eau chaude, les feuilles de gélatine une à une en tournant à la spatule. \\
Passer la compote et la verser dans une terrine. Y ajouter la gélatine délayée et battre vigoureusement au fouet 15 à 20 minutes. \\
Ca donne un mélange ferme et blanc neigeux. \\
Verser dans un moule à charlotte caramélisé ou dresser en dôme dans une patte foncée avec une spatule trempée d’eau chaude. \\
Faire un caramel dans un poêlon. Quand il atteint la teinte recherchée, y ajouter $\nicefrac{1}{2}$ verre d’eau. Reprendre l’ébullition du caramel, sinon la mixture sera trop épaisse.\\
En verser un peu sur le dôme de compote obtenu et servir le reste dans une saucière.\\
\\

\end{minipage}

\begin{minipage}[c]{\textwidth}
\recette{Princesse cake }
\otor{Tilou}
\ingredients{
\item 6 cuillères de farine
\item 6 cuillères de sucre
\item 6 cuillères de lait
\item 2 œufs entiers
\item 125 de beurre ramolli
}
Cuisson : 20 minutes à four moyen\\
\\

\end{minipage}

\begin{minipage}[c]{\textwidth}
\recette{Petits four aux blancs d’œufs}
\ingredients{
\item 30 à 40 petits fours.
\item 2 blancs d’œufs
\item 90 gr de sucre en poudre 
\item Un zeste de citron râpé finement
}
Mélanger le tout au batteur (blancs non battus en neige à l’avance).\\
Huiler une feuille de papier blanc, poser sur une tôle, en grosses gouttes (1/2 noix). \\
Y laisser anuber le mélange à 3 minutes d’intervalles\\
Cuire à four doux 15 à 20 minutes. Découler du papier et conserver dans une boite en fer.\\
\\

\end{minipage}

\begin{minipage}[c]{\textwidth}
\recette{Crème à la glace }
\otor{Odette}
\ingredients{
\item 1 boule de lait « évaporé »
\item 2 cuillères de sucre fariné
\item 2 jaunes d’œufs
\item OU
\item 1 boite de lait condensé 
\item 1/3 tiers de lait (grandeur de la boite de lait évaporé)
\item 2 jaunes d’œufs 
\item Ajouter crème fraiche à volonté 
}
Parfums : \\
2 cuillères à soupe de Nescafé \\
2 cuillères à soupe de confiture\\
Sucre vanillé\\
\\

\end{minipage}

\begin{minipage}[c]{\textwidth}
\recette{Flan caramel}
1. Confection du caramel :\\
Chauffer 6 cuillères à soupe de sucre et $\nicefrac{1}{2}$ verre d’eau à feu vif. \\
Remuer jusqu’à ce que la mousse soit très blonde. \\
Verser alors dans le flotter terre et l’étendre en tournant\\
2. Battre 5 œufs entiers avec 5 cuillères à soupe de sucre. \\
Faire bouillir $\nicefrac{3}{4}$ d’une pinte de lait, puis y verser les œufs.\\
3. Verser le tout sur le caramel\\
4. Cuire à four chaud, baissé à $\nicefrac{1}{2}$ dans la plaque + eau bouillante pendant 25 minutes !\\
\\

\end{minipage}

\begin{minipage}[c]{\textwidth}
\recette{Dartois à la crème d’amande }
Etaler 2 rectangles de pâtes feuilletés (30 cm x 10-12 cm). L’un des rectangles mince, l’autre le double de volume.\\
Poser la bande mince sur la toile à pâtisserie, étaler une bonne couche de crème d’amande (recette précédente), humecter les bords à l’eau froide et y coller la deuxième bande de pâte. \\
Pour la deuxième bande de pâte, boucher les bords avec les doigts. \\
Dorer à l’œuf. Rayer avec la pointe d’un couteau et piquer avec l’aiguille sur toute l’épaisseur. \\
Cuire au four 20 à 30 min.\\
Saupoudrer de sucre glace et doré au four quelques minutes (surveiller)\\
\\

\end{minipage}

\begin{minipage}[c]{\textwidth}
\recette{Le phare du Finistère}
Mélanger 250g de sucre en poudre à 4 œufs entiers. Délayer avec un Litre de lait chaud.  Ajouter 125 g de raisins et 1 verre à liqueur de Rhum. \\
Enfourner dans un plat bien beurré environ 1 heure. \\
A manger chaud ou froid !\\
\\

\end{minipage}

\begin{minipage}[c]{\textwidth}
\recette{Strudel}
\ingredients{
\item Pâte : 
\item 250 g de farine
\item 1 œuf
\item 2 cuillères à soupe d’huile
\item Quelques cuillerées de lait
\item 1 cuillère à café de sel
\item Garniture :
\item 1 kg de pomme
\item 150g de sucre
\item 100g de beurre
\item 100g de chapelure brune
\item 50g de raisins secs
\item 50g d’amandes
\item 1 cuillère à soupe de cannelle en poudre
}
1. La pâte\\
Dans la terrine en fontaine, mettre la farine, le sel, l’huile et l’œuf. Travailler avec les doigts en ajoutant peu à peu le lait pour avoir une pâte mole. Travailler cette dernière 10 minutes avec la paume de la main jusqu’à ce qu’elle soit bien unie. \\
En faire une boule et laisser reposer une heure dans un torchon humide. \\
2. La garniture\\
Peler et couper les pommes en tranches  (plutôt petites lamelles).\\
Les mettre dans une terrine avec environ $\nicefrac{1}{2}$ sucre (75g), y faire fondre le beurre et y ajouter la chapelure. \\
Remuer quelques instants avec la cuillère en bois. Tout le beurre doit être absorbé. \\
Garder un peu au chaud. \\
Nettoyer les raisins et les gonfler dans un peu d’eau chaude. \\
3. Etendre la pate d’abord au rouleau puis l’étirer sur un drap jusqu’à ce qu’elle soit très fine, en badigeonnant au pinceau trempé dans de l’huile pour éviter qu’elle sèche ou casse. Obtenir une grande surface, environ 1m carré très fine comme du papier. \\
Mettre des pommes à une des extrémités. \\
Saupoudrer de chapelure cuite et de beurre puis de raisons, d’amandes et de cannelle et le reste du sucre.\\
Prendre la pâte dans les deux mains au dessous et faire rouler la garniture jusqu’au bout. On a fait ainsi un grand chausson. \\
Dorer à l’œuf (jaune) et glisser à four chaud sur la plaque à pâtisserie environ $\nicefrac{1}{2}$ heure. \\
Trois minutes avant la fin, saupoudrer de sucre et faire caraméliser en surveillant car cela brûle très vite ! A servir chaud, tiède ou froid.\\
\\

\end{minipage}

\begin{minipage}[c]{\textwidth}
\recette{Tuiles }
\ingredients[(pour 4 personnes)]{
\item 100 gr de beurre fondu
\item 100 gr de sucre glace
\item 5 cuillerées à soupe de jus d’orange
\item 1 cuillerée de Grand Marnier
\item 100 gr d’amandes effilées
\item 30 gr de farine
\item Le zeste râpé d’une orange
\item 1 pincée de sel
}
Mélanger beurre, sucre et le zeste d’orange, puis incorporer le jus d’orange et le Grand Marnier. Ajouter la farine et le sel. \\
Laisser reposer 2 heures au frigo. \\
Allumer le four à thermostat 5 (environ 150 degrés). Sur une plaque anti adhésive ou du papier sulfurisé, disposer de petits tas de pâte. Les étaler avec les doigts pour leur donner une forme ovale, puis les parsemer d’amandes. \\
Faire cuire au four et retirez dès que la pâte blondit. Laisser tiédir. \\
Décoller les tuiles avec une spatule et les disposer sur un rouleau à pâtisserie ou une bouteille en verre. \\
Laisser refroidir, garder au frais dans un endroit sec. \\
Servir avec les oranges (ci dessous)\\
\\

\end{minipage}

\begin{minipage}[c]{\textwidth}
\recette{Salade d’oranges }
\otor{Vivette}
Couper à vif une orange par personne en fines tranches. \\
Faire un sirop de sucre : environ 100 gr de sucre et 2 cuillères à soupe d’eau sur feu doux en tournant très souvent. Aromatiser avec de la badiane (ou anis étoilé). \\
Verser sur les oranges.\\
\\

\end{minipage}

\begin{minipage}[c]{\textwidth}
\recette{Charlotte aux pommes}
\ingredients{
\item 800 gr de pommes
\item 400 gr de sucre
\item 3 verres à moutarde d’eau
\item 1 bâton de vanille
\item Le jus d’1 citron
\item Calvados
\item Boudoirs (« biscuits »)
\item Crème fraiche
}
Peler les pommes et enlever l’intérieur. Couper les morceaux. Ajouter 300 gr de sucre, les 3 verres à moutarde d’eau, le jus de citron et le bâton de vanille. Cuire le tout très doucement pendant 40 minutes. \\
Garder un verre d’eau. Y ajouter 100 gr de sucre et faire un sirop. Refroidir puis ajouter un décilitre de calvados. \\
Beurrer un moule à charlotte. Tremper les 36 biscuits à la cuillère dans le sirop et les ranger bien serrés dans le moule (fonds et bords). \\
Dans le fond, ajouter une couche de pomme et le reste du calvados, une couche de biscuits trempés, une couche de pomme, une couche de biscuits… jusqu’au dessus. \\
Poser une petite assiette avec un poids dessus. Mettre au frigo. \\
Garnir de crème fraiche avant de servir !\\
\\

\end{minipage}

\begin{minipage}[c]{\textwidth}
\recette{Gâteau aux pommes }
Peler et épépiner 250 gr de pommes. Cuire au beurre rapidement -> sucre (un peu). ?? Ajouter 40 gr de sucre, 40 gr de beurre, 1 jaune d’œuf, 60 gr de farine, $\nicefrac{1}{2}$ paquet de levure sèche et le blanc d’œuf battu en neige. Mélanger et parfumer avec des zestes d’orange ou de citron râpé. \\
Cuire 30 à 40 minutes dans une cocotte fermée et beurrée à four doux.\\
On met une cocotte dans le four ? \\
\\

\end{minipage}

\begin{minipage}[c]{\textwidth}
\recette{Truffe aux noisettes}
Prendre du chocolat (250 gr) râpé ou fondu doucement avec une cuillère d’eau et le mélanger avec 250 gr de sucre en poudre.\\
Griller et râper à la moulinette 250 gr de noisettes et mélanger au mélange choco et sucre. Ajouter 150 gr de beurre ramolli, 1 paquet de sucre vanillé. Si le mélange est trop épais, ajouter 1 ou 2 cuillères de lait. \\
Reposer au frais 1 à 2 heures. \\
Former des boules avec les mains et rouler dans le choco râpé ou granulé. \\
Conserver au frais jusqu’au moment de servir.\\
\\

\end{minipage}

\begin{minipage}[c]{\textwidth}
\recette{Truffe de l’ogre}
Passer à la moulinette 200 gr de biscuits à la cuillère. Y incorporer 250 gr de choco râpé, 200 gr d’amandes râpées, 200 gr de sucre fin, 250 gr de beurre mou, 2 jaunes d’œufs, 1 cuillère à café d’essence de café et 2 cuillères à dessert de whisky. \\
Former des boules et les rouler dans le choco râpé ou granulé. \\
Garder au frais jusqu’au moment de déguster. \\
\\

\end{minipage}

\begin{minipage}[c]{\textwidth}
\recette{Truffe au chocolat}
Mélanger 250 gr de beurre avec 125 gr de cacao. Malaxer avec une cuillère en bois. \\
Sirop : $\nicefrac{1}{2}$ cuillère de sucre et $\nicefrac{1}{2}$ verre d’eau. Cuire 7 à 10 minutes jusqu’à ce que ca devienne du sirop (une grosse goutte ronde). \\
Verser peu à peu le sirop de cacao et le beurre en remuant bien. Ajouter 2 petits verres de liqueur. Battre à la cuillère en bois.\\
Mettre au froid. \\
Faire des boules et les rouler dans le cacao. \\
\\

\end{minipage}

\begin{minipage}[c]{\textwidth}
\recette{Tarte alsacienne}
1. Prendre 500g de cerises rouges transparentes, les laver, enlever la queue et les noyaux.\\
2. Préparer la même crème que pour le clafoutis : deux œufs, une cuillère à café de farine, deux cuillères à soupe de sucre en poudre, 70 gr de crème épaisse (ou beurre), un verre de lait et un petit verre de kirsch.\\
3. Préparer une pâte brisée :\\
150 gr de farine, 80 gr de beurre, 1/3 de verre d'eau et une pincée de sel.\\
Malaxer du bout des doigts dans une terrine tout ces éléments jusqu’à obtenir une boule de pâte que l’on abaisse au rouleau à une épaisseur de 3 mm.\\
4. Garnir un moule à tarte beurré : Poser les cerise sur la pate, puis y verser la crème. Porter au four chaud pendant 35 minutes, puis laisser refroidir et démouler.\\
\\

\end{minipage}

\begin{minipage}[c]{\textwidth}
\recette{Tarte au Reine-Claude}
\ingredients{
\item 150 gr de farine
\item 75 gr de beurre
\item Un peu de lait
\item 1 cuillère à café de bicarbonate
\item 1 pincée de sel
\item 1 cuillère à soupe de sucre
}
Etendre la pâte à la main dans la platine, la piquer à la fourchette.\\
Y placer un livre de Reine-Claudes coupées, la tranche en l'air.\\
Sucrer à la sortie du four.\\
\\

\end{minipage}

\begin{minipage}[c]{\textwidth}
\recette{Crème caramel }
Faire bouillir le litre de lait avec 75 gr de sucre.\\
Ajouter au lait bouillant un caramel formé en ajoutant un filet d’eau à 100 gr de sucre et en l’ayant laissé cuire jusqu’à brunissement. \\
Ajouter 3 cuillerées de custard préalablement délayée à l’eau froide.\\
Laisser un peu refroidir dans le poêlon avant de verser dans un plat.\\
\\

\end{minipage}

\begin{minipage}[c]{\textwidth}
\recette{Tarte Normande }
\ingredients[(2 petites tartes)]{
\item 250 gr de farine. 
\item 1 œuf entier battu.
\item 15 gr de levure délayée dans de l’eau tiède.
\item 25 gr de sucre (environ 1 cuillerée à soupe)
\item 25 gr de beurre (environ 1 cuillerée à soupe)
}
Faire un puits avec la farine. \\
Mettre un peu de sel sur les bords du puits.\\
Mélanger le beurre et le sucre dans le trou du puits.\\
Ajouter l’œuf et la levure.\\
Pétrir jusqu’à obtenir une pâte lisse.\\
Diviser la pâte en deux et la laisser monter à chaleur douce le temps de la préparation de la garniture (fruits ou sucre + œuf battu et beurre).\\
Abaisser la pâte au rouleau. La mettre dans la platine graissée. La garnir.\\
Cuisson environ 20 minutes au four pas trop chaud.\\
\\

\end{minipage}

\begin{minipage}[c]{\textwidth}
\recette{Crème pâtissière}
Tourner en crème avec un fouet 3 jaunes d’œufs et 125 gr de sucre (environ 5 cuillerées) jusqu’à ce que ça « cloque » (environ 10 minutes).\\
D’autre part, délayer 3 cuillerées de maïzena avec un peu de lait froid.\\
Mélanger les deux préparations et verser les dans du lait bouillant en fouettant jusqu’à épaississement.\\
\\
Pour la crème à choux, fouetter la crème pâtissière jusqu’à refroidissement dans l’eau froide pour éviter les peaux.\\
\\

\end{minipage}

\begin{minipage}[c]{\textwidth}
\recette{Crêpes à la Paul Bouillard}
\ingredients{
\item 125 gr de farine.
\item 3 œufs battus au fouet.
\item 1 cuillerée de sucre.
\item 1 pincée de sel.
\item Du lait (à ajouter jusqu’à ce que la pâte soit bien liquide)
\item 1 forte cuillerée à soupe de beurre fondu.
\item 1 cuillerée de cognac, rhum ou kirsch (facultatif).
}
Mélanger le tout.\\
\\

\end{minipage}

\begin{minipage}[c]{\textwidth}
\recette{Fromage blanc en crème }
Bien battre 2 pots de fromage blanc.\\
Sucrer.\\
Ajouter 1 jaune d’œuf.\\
Mélanger et servir.\\
\\

\end{minipage}

\begin{minipage}[c]{\textwidth}
\recette{Pommes Bernoises}
Cuire les pommes de terre entières 5 minutes à l’eau bouillante.\\
Les égoutter et les passer à l’eau froide. \\
Saler.\\
Couper les pommes de terre en tranches et en garnir le fond d’un plat à gratin. \\
Ajouter du beurre fondu et couvrir de fromage (emmental ou gruyère) coupé en lamelles.\\
Cuire au four jusqu’à ce que ce soit doré.\\
\\

\end{minipage}

\begin{minipage}[c]{\textwidth}
\recette{Pâte feuilletée}
Faire un puits avec 250 gr de farine.\\
Ajouter 50 gr de margarine (environ $\nicefrac{1}{4}$ de paquet).\\
1 pincée de sel et de l’eau.\\
Mélanger la pâte en la travaillant d’abord du bout des doigts puis du poignet jusqu’à ce qu’elle soit bien lisse.\\
Rouler la pâte au rouleau à tarte, bien l’étendre.\\
Lui donner une forme rectangulaire. Encore bien la rouler. \\
La plier en trois dans les deux sens (-> 9).\\
La rouler au rouleau de nouveau. Lui redonner une forme rectangulaire.\\
Replier en 9.\\
La laisser reposer 15 minutes au froid.\\
Incorporer 50 gr de beurre ou de margarine.\\
Plier la pâte en 9 puis encore en 9. La plier en 9 puis encore en 9.\\
(On a finalement incorporé 200 gr de corps gras).\\
\\

\end{minipage}

\begin{minipage}[c]{\textwidth}
\recette{Wiener Torte }
\ingredients{
\item 100 gr de beurre.
\item 100 gr de chocolat râpé.
\item 100 gr de sucre.
\item 75 gr de chapelure blanche.
\item 5 œufs.
}
Dans une terrine, travailler le beurre déjà mou. \\
Ajouter le sucre, le chocolat. Travailler le tout à l’aide d’une spatule jusqu’à obtenir une mousse homogène. \\
Ajouter un à un les jaunes d’œufs. \\
Ajouter la chapelure.\\
Ajouter les blancs d’œufs battus en neige très ferme. \\
Beurrer un moule à tarte à bords hauts. Y poser dans le fond un papier beurré. \\
Verser la masser fluide.\\
Cuire à four doux 35 à 40 minutes. \\
\\

\end{minipage}

\begin{minipage}[c]{\textwidth}
\recette{Gâteau de 20 ans}
\ingredients{
\item 250 gr de chocolat râpé.
\item 250 gr de sucre en poudre.
\item 250 gr d’amandes épluchées.
\item 2 jaunes d’œufs.
\item 2 œufs entiers + 1 blanc d’œuf. 
}
Casser 2 œufs entiers dans une terrine.\\
Ajouter le sucre en poudre et mélanger à l’aide d’une spatule. \\
Ajouter le chocolat et mélanger.\\
Ajouter les amandes qui auront été préalablement jetées 1 minute dans de l’eau bouillante puis décortiquées et moulues.\\
Une fois que la pâte est homogène et épaisse, y incorporer en mélangeant à l’aide d’une fourchette les blancs battus en neige très ferme.\\
Verser la pâte dans un moule beurré avec un papier dans le fond.\\
Cuire à feu doux.\\
Laisser un peu reposer. \\
Détacher à l’aide d’un couteau.\\
\\

\end{minipage}

\begin{minipage}[c]{\textwidth}
\recette{Tourte Viennoise }
\otor{Tilou}
Casser 3 œufs en séparant les blancs des jaunes dans deux bols différents.\\
Ajouter 180 gr de sucre aux jaunes d'oeufs. Battre et jeter d’un seul coup 40 gr de farine. Ajouter une cuillerée à café d’eau. Fouetter encore 2 minutes.\\
Fouetter à part les blancs en neige ferme.\\
Mettre chauffer 180 gr de chocolat avec une cuillerée d’eau. Ajouter au chocolat fondu 180 gr de beurre.\\
Mélanger le mélange à base de chocolat à celui à base des jaunes d’œufs. \\
Ajouter les blancs en neige. \\
Verser le tout dans un moule beurré de 25 cm de diamètre et 5 cm de hauteur et enfourner dans un four très chaud en baissant la flamme de moitié.\\
Cuisson 25 minutes.\\
Défourner. Laisser le gâteau retomber. Le démouler après 30 minutes.\\
Ne manger que le lendemain. \\
\\

\end{minipage}

\begin{minipage}[c]{\textwidth}
\recette{Gâteau Anversois}
\ingredients{
\item 150 gr de farine.
\item 100 gr de fécule.
\item 200 gr de sucre.
\item 4 cuillerées de lait
\item 1 œuf (le blanc battu en neige). 
\item 100 gr de beurre tiède.
\item 30 gr de cacao en poudre.
\item 1 paquet de sucre vanillé.
\item 1 paquet de levure.
}
Mélanger le tout.\\
\\
Enfourner dans un four tempéré dans un moule à charnière beurré et avec un peu de farine.\\
\\
Glaçage :\\
Dans une casserole faire fondre au bain-marie 2 tablettes de chocolat. \\
1 bon morceau de beurre ($\nicefrac{1}{2}$ cuillère à soupe).\\
1 de sucre faune = 2 cuillères d’eau chaude ??\\
Bouillir.\\
Le mélange final doit être épais et mat.\\
Couler le mélange sur le gâteau.\\
\\

\end{minipage}

\begin{minipage}[c]{\textwidth}
\recette{Crème Anglaise}
Dans un bol battre 2 œufs, 2 cuillères à soupe de sucre, 20 gr de farine et un peu de lait ($\nicefrac{1}{2}$ cuillère à soupe).\\
Y verser un verre de lait (bouillant) parfumé soit avec de la vanille soit avec du cacao ou encore avec $\nicefrac{1}{2}$ cuillère à café d’essence de café et continuer à battre. \\
Verser le tout dans une casserole.\\
Un bouillon très bref.\\
Verser dans une terrine.\\
Laisser refroidir.\\
\\

\end{minipage}

\begin{minipage}[c]{\textwidth}
\recette{Crème Saint-Honoré}
Il suffit de verser dans la crème anglaise finie une feuille de gélatine fondue et battre 2 blancs d’œufs à ajouter au mélange avec une fourchette. Mettre le tout à refroidir.\\
\\

\end{minipage}

\begin{minipage}[c]{\textwidth}
\recette{Soufflé au pain}
Laisser tremper de la mie de pain dans du lait chaud. \\
Bien égoutter puis écraser à la fourchette.\\
Y mélanger 2 jaunes d’œufs, 100 gr de sucre en poudre et 100 gr de beurre fondu.\\
Battre 2 ou 3 blancs d’œufs en neige. L’incorporer à la préparation. \\
Verser le tout dans un plat beurré.\\
Mettre au four. Le soufflé doit devenir bien doré. \\
Servir chaud.\\
Napper de sauce abricot parfumée au kirsch (facultatif).\\
\\

\end{minipage}

\begin{minipage}[c]{\textwidth}
\recette{Crêpes}
\ingredients{
\item 600 gr de farine. 
\item 230 gr de beurre ou de margarine.
\item 4 œufs.
\item 100 gr de cassonade.
\item 100 gr de sucre blanc.
\item 1 litre de lait. 
}
Faire un puits avec la farine.\\
Ajouter une pincée de sel et les jaunes d’œufs. Mélanger avec une spatule.\\
Ajouter le beurre ou la margarine fondu.\\
Ajouter le lait.\\
Ajouter les blancs battus en neige.\\
Le résultat doit être une pâte liquide masquant le dos de la cuillère.\\
\\

\end{minipage}

\begin{minipage}[c]{\textwidth}
\recette{Riz dessert }
\ingredients[(2 personnes)]{
\item $\nicefrac{1}{2}$ litre de lait.
\item 75 gr de riz.
\item 1 ou 2 jaunes d’œufs.
\item 2 cuillères à soupe de sucre.
\item 1 morceau de beurre. 
}
Faire bouillir le lait. Y ajouter le riz.\\
Laisser cuire 30 minutes. \\
Ajouter le sucre, le jaune, le beurre et une goutte d’essence d’amande.\\
Garnir de geler de groseilles.\\
\\

\end{minipage}

\begin{minipage}[c]{\textwidth}
\recette{Bombe Marocaine}
Verser du riz dessert dans un bol beurré. \\
Laisser bien épaissir.\\
Pour démouler passer le bol au bain-marie.\\
Y verser un ?? au chocolat.\\
\\

\end{minipage}

\begin{minipage}[c]{\textwidth}
\recette{Glacé au chocolat}
Faire fondre une ?? de chocolat au bain-marie.\\
Y ajouter 1 cuillère à soupe d’eau chaude, $\nicefrac{1}{2}$ cuillère à soupe de sucre et 1 morceau de beurre.\\
Mélanger à feu vif.\\
Faire bouillir le mélange tout en tournant jusqu’à épaississement.\\
Verser sur la bombe.\\
\\

\end{minipage}

\begin{minipage}[c]{\textwidth}
\recette{Praliné}
\ingredients{
\item 50 gr d’amandes.
\item 60 gr de sucre farine. 
}
Mettre les amandes et le sucre dans un poêlon sur feu doux en mélangeant bien à la fourchette jusqu’à début d’épaississement. Attention à ne pas laisser bruler.\\
Une fois cuit, verser le mélange sur une plaque préalablement huilée. \\
Laisser refroidir.\\
Garder le praliné dans une boite en fer.\\
Avant de l’employer pour un gâteau ou tout autre dessert, le moudre avec un moulin.\\
\\

\end{minipage}

\begin{minipage}[c]{\textwidth}
\recette{Croute aux fraises}
Ajouter à la pâte ci-dessus expliquée deux œufs. \\
Mettre la pâte dans un moule. La cuire à part, sans la garnir de fruit, en couvrant d’un papier cuisson avec des pois.  \\
Garnir de fraises et de sucre au moment de servir.\\
\\

\end{minipage}

\begin{minipage}[c]{\textwidth}
\recette{Soufflé au Grand Marnier}
Pâte : Faire fondre 40 gr de beurre dans une casserole. Ajouter 40 gr de farine, un bon verre de lait et 40 gr de sucre. Tout en remuant la préparation au fouet, la porter à ébullition. Laisser tiédir.\\
Battre 5 jaunes d’œufs. Y incorporer les 5 blancs préalablement préalablement battus en neige bien ferme. Ajouter un petit verre de Grand Marnier. \\
Beurrer un moule. Le saupoudrer de sucre. Y mouler la pâte puis y verser le mélange œufs-Grand Marnier. Cuire à four modéré (max 200\degrees) pendant 25 minutes. \\
\\

\end{minipage}

\begin{minipage}[c]{\textwidth}
\recette{Tarte aux groseilles}
Pâte :\\
\\
250 gr de farine.\\
125 gr de beurre.\\
10 cl de lait.\\
$\nicefrac{1}{2}$ cuillère à café de bicarbonate.\\
\\
Faire un puits avec la farine. En son centre, y mettre tous les autres ingrédients. Ramener la farine avec deux doigts sur les autres ingrédients. Mélanger. \\
\\
Crème :\\
\\
2 œufs.\\
20 cl de lait.\\
100 gr de sucre.\\
1 cuillère à soupe de farine.\\
\\
Battre les deux œufs entiers. Ajouter la farine. Mélanger.\\
Faire bouillir le lait avec le sucre. Verser ce lait sucré sur le mélange œufs-farine.\\
Remettre le tout sur le feu. Fouetter jusqu’à l’obtention d’une crème relativement épaisse. Retirer du feu et laisser tiédir.\\
\\
Sur la pâte moulée dans le moule, étendre la crème. Garnir de fruits. \\
NB : Pour des groseilles rouges, les blanchir préalablement avec un peu de sucre.\\
\\
Mettre le tout au four (150 et 200\degrees). Cuisson entre 25 et 30 minutes.\\
\\

\end{minipage}

\begin{minipage}[c]{\textwidth}
\recette{Gâteau aux pommes (culte)}
250 gr de beurre\\
125 gr de sucre\\
Mélanger le tout. Ajouter 3 œufs entiers bien battus. \\
Verser par cuillérée 250 gr de farine et une cuillère à café de levure.\\
Verser la pâte dans un moule à cheminée préalablement beurré. Décorer de pommes découpées en quartier et tranchées (la partie ronde contre la cheminée) et cuire à feu doux.\\
\\

\end{minipage}

\begin{minipage}[c]{\textwidth}
\recette{Croute aux fraises (pâte croquante)}
\ingredients{
\item 250 gr de farine. 
\item 125 gr de sucre.
\item 1 œuf entier et 1 jaune d’œuf.
\item 1 pincée de sel.
\item 200 gr de beurre ramolli.
}
Faire une fontaine avec la farine. Mettre le beurre en son centre. L’incorporer à la farine avec deux doigts. Ajouter les œufs puis le sucre. Mélanger jusqu’à l’obtention d’une pâte bien lisse. Bouler la pâte, la saupoudrer de farine. Laisser reposer 15 minutes. \\
Mouler mince la pâte. Couvrir la pâte de papier cuisson avec des pois. Cuire à feu doux (125 – 150\degrees), la chaleur doit venir du dessus, pendant 20 – 25 minutes en surveillant bien. \\
Enlever les pois. \\
Pour dorer le dessus de la pâte, cuire pendant 5 minutes supplémentaires, la chaleur doit toujours venir du dessus.\\
Pour que la pâte soit bien croquante, la faire un jour à l’avance. La garder au frais. Le lendemain, avant de servir, la garnir de fraises et de sucre.\\
\\

\end{minipage}

\begin{minipage}[c]{\textwidth}
\recette{Clafoutis aux cerises}
1 kg de cerises rouges transparentes lavées, dénoyautées et équeutées. Les mettre dans un plat en terre préalablement beurré. \\
Dans une terrine à part, battre 2 œufs entiers et 1 jaune. Ajouter et mélanger une cuillère à soupe de farine. Ajouter et mélanger 150 gr de sucre en poudre. Ajouter et mélanger un bon morceau de beurre. Ajouter et mélanger un verre de lait et une pincée de sel. \\
Verser la préparation sur les cerises. \\
Préchauffer le four à 250\degrees. En enfournant, baisser la température de moitié. Laisser cuire jusqu’à ce que la crème se soit solidifiée (au moins 30 minutes).  \\
Sortir le plat du four et le saupoudrer de sucre.\\
Laisser refroidir avant de servir.\\
\\

\end{minipage}

\begin{minipage}[c]{\textwidth}
\recette{Soupe aux cerises}
\ingredients{
\item 1 kg de cerises noires pas trop grosses lavées et équeutées. Ne pas sortir les noyaux.
}
Dans une casserole, faire fondre 60 gr de beurre. Ajouter 1 cuillère à soupe de farine. Mélanger. Mouiller la préparation avec un verre de vin rouge et un verre d’eau. Mélanger et laisser bouillir. Ajouter les cerises quand ça bout. Réduire le feu et mélanger délicatement. Ajouter 3 cuillères à soupe de sucre en poudre et quelques pincées de cannelle. \\
Mettre le couvercle et laisser cuire à petit feu 20 minutes.\\
Ajouter deux petits verres de kirsch. Faire bouillir 3 minutes.\\
Sur le côté, faire frire dans du beurre des tranches de pain. Les mettre dans le fond de la soupière puis y verser les cerises. Servir chaud. \\
\\

\end{minipage}

\begin{minipage}[c]{\textwidth}
\recette{Gâteau 4/4 }
\ingredients{
\item 250 gr. de farine 
\item 250 gr. de sucre
\item 250 gr. de beurre
\item 250 gr. d’œufs (c'est-à-dire 4 œufs). Séparer les jaunes des blancs et battre ces derniers en neige. 
\item $\nicefrac{1}{2}$ paquet de levure.
}
Mélanger le tout et cuire à feu doux !\\
\\

\end{minipage}

\begin{minipage}[c]{\textwidth}
\recette{Brioche aux pommes}
Mélanger 250g de beurre avec 150g de sucre, puis 3 oeufs entiers bien battus. Versez alors 250 g de farine et 3 cuillères de lait et une cuillère à café de levure.  Déposez dans un moule à cheminée, glissez à four doux décoré de deux pommes coupées en quartiers et tranchées.\\
\\

\end{minipage}

\begin{minipage}[c]{\textwidth}
\recette{Gâteau Claude}
\ingredients{
\item Farine 250 gr.
\item Sucre 150 gr. 
\item 4 cuillères à soupe de lait froid .
\item Beurre tiède 100 gr. 
\item 250 gr. d’œufs (c'est-à-dire 4 oeufs). Séparer les jaunes des blancs et battre ces derniers en neige. 
\item $\nicefrac{1}{2}$ paquet de levure.
\item Une cuillerée de liqueur ou de sucre vanillé. 
\item Une poignée de raisins secs.
}
Cuire à four modéré.\\
\\

\end{minipage}

\begin{minipage}[c]{\textwidth}
\recette{Savoie}
Mélanger dans une terrine 250 gr. de sucre et 7 jaunes d’œufs. Garder les blancs pour plus tard.\\
Travailler avec le fouet en ajoutant un paquet de sucre vanillé, puis petit à petit, 100 gr de farine et 100 gr de fécule.\\
Battre en neige ferme $\nicefrac{1}{4}$ des 7 blancs d’œufs. Quand c’est un peu détendu, verser le reste des blancs avec souplesse. Mettre le mélange dans un moule beurré jusqu’au $\nicefrac{3}{4}$ de sa hauteur. \\
Cuire à feu très doux et régulier. Quand c’est cuit, laisser reposer dans le moule quelques minutes puis démouler sur une grille.\\
\\

\end{minipage}

\begin{minipage}[c]{\textwidth}
\recette{Beignets au platte kaas (fromage blanc)}
Mélanger 2 jaune d’œufs, 50 gr de sucre, 100 gr de farine et 250 gr de platte kaas.\\
Ajouter 2 blancs en neige ferme.\\
Mettre une cuillère à café à la fois dans la friture.	\\
\\

\end{minipage}

\begin{minipage}[c]{\textwidth}
\recette{Mousse au chocolat }
Faire fondre au bain marie une le chocolat. \\
Dans une terrine mélanger un jaune d’œufs avec une cuillère à soupe de sucre jusqu’à ce que ça cloque ! \\
Battre le blanc en neige ferme.\\
Incorporer 25 à 30 gr. de beurre au chocolat fondu.\\
Ajouter le jaune et le sucre au mélange.\\
Ajouter le blanc en neige avec une fourchette et l’incorporer légèrement pour ne pas briser le blanc. \\
Mixer le tout dans le bol.\\
\\

\end{minipage}

\begin{minipage}[c]{\textwidth}
\recette{Tarte aux reines-claudes}
\ingredients{
\item 150g de farine
\item 75g de beurre
\item un peu de lait
\item une cuillère à café de baking
\item une pincé de sel
\item une cuillère à soupe de sucre
\item un litre de reine claude
}
préparation :\\
Etendre la pâte à la main dans la platine et la piquer à la fourchette. Y placer les reines-claudes coupées la tranche en l'air. Mettre au four et sucrer à la sortie du four.\\
\\

\end{minipage}

\begin{minipage}[c]{\textwidth}
\recette{Kugelhopf alsacien}
Dans une terrine, travaillez 150g de beurre avec 150g de sucre et 6 jaune d'œufs jusqu'à ce que ce soit bien lisse. Incorporez y 180g de farine, 9 blancs d'oeufs en neige et les zeste d'un citron finement haché. Versez dans le moule à kugelhopf. Cuire à four chaud 3/4 d'heure. En sortant le gâteau du four, saupoudrez le de sucre en poudre et laissez le refroidir.\\
\\

\end{minipage}

\begin{minipage}[c]{\textwidth}
\recette{Tarte viennoise aux pommes}
Dans un récipient, verser 1/3 de verre d'eau froide et 160g de farine. Peu à peu, incorporer 80g de beurre, deux pincées de sel (si la pate colle un peu ajoutez y 1 cuillère de farine). En faire une boule que vous laissez reposer 1/4 d'heure dans une serviette. Roulez la pate et en garnir une platine beurrée. Piquez la pate avec une fourchette\\
La garnir de compote. Pour la compote : découper 150g de pomme du Canada et les mettre à cuire avec 4 cuillères de sucre .\\
Couper en quartier deux pommes crues et les disposer sur la compote.\\
Saupoudrer le tout d'une demie cuillère à café de cannelle, une cuillère de sucre et deux cuillère de farine. Enfourner 25 min\\
\\

\end{minipage}

\begin{minipage}[c]{\textwidth}
\recette{Tarte aux pruneaux}
La veille, tremper les pruneaux. Dénoyauter-les le lendemain\\
Faire une pâte à tarte avec 180g de farine, 90g de beurre, 1/3 de verre d'eau, 1 pincé de sel.\\
Couvrir la pâte mise dans la platine d'une crème faite avec deux jaunes d’œufs etun œuf entier, 4 cuillère de sucre et une de farine, 125g de crème, 1 verre de lait et 1 verre de kirsch.\\
Garnir de pruneaux ouverts. Cuire 35 à 40 min. Arrosez la tarte de kirsch chauffé et enflammer en servant\\
\\

\end{minipage}

\begin{minipage}[c]{\textwidth}
\recette{Puits d'amour}
Battre deux jaunes d'oeufs avec du sucre jusqu'a obtenir la consistance d'une mayonnaise. Remplir 5 moules du mélange, verser une cuillère de rhum et ajouter une couche de confiture de groseille et une couche de banc d'oeuf en neige sucré.\\
\\

\end{minipage}

\begin{minipage}[c]{\textwidth}
\recette{Soufflé à l'orange}
\ingredients{
\item trois oranges
\item 1 citron
\item 6 blancs d'œufs
\item 150g de sucre
}
Préparation : râper les zestes, battre les blancs en neige et y incorporer les zeste et le sucre.\\
Garnir un moule avec la préparation et y ajouter du sucre. Mettre 20 à 30 minutes à four doux.\\
servir le soufflé chaud.\\
\\

\end{minipage}

\begin{minipage}[c]{\textwidth}
\recette{Kugelhopf (bouillard)}
\ingredients{
\item 500g de farines
\item 200g de sucre
\item 12g de sel
\item 35g de levure
\item 3 oeufs
\item 200g de raisin sec
\item 150g de beurre
\item 1/10 de litre de lait
}
Faire un levain :\\
Mélangez la levure à 1/2 tasses à déjeuner de lait ou d'eau tiède. Versez au centre de la farine. Mélangez jusqu'à obtenir une pâte assez fluide. Recouvrir de farine et laisser monter jusqu'au rejet de la farine. Mélangez le restant de farine avec les oeufs, le sucre le sel et un peu de lait. Pétrissez et terminez par le levain. Mélangez le beurre et les raisins bien épongés. Mettez en moule pour le laisser monter. Enfournez comme le pain.\\
Cuire une heure au four à chaleur modérée.\\
\\

\end{minipage}

\begin{minipage}[c]{\textwidth}
\recette{Confiture d'orange}
Faire une julienne des zestes d'orange avec 6 oranges et 3 citrons épluchés très fins et coupés, le tout couvert de 3 L d'eau par kilo de fruit jusqu'au lendemain.\\
Faire alors évaporer le tiers et mettre par kilo restant 1200g de sucre. Faire cuire.\\
\\

\end{minipage}

\begin{minipage}[c]{\textwidth}
\recette{Salade de fruit}
\ingredients{
\item Deux oranges en tranche
\item Deux bananes en rondelle
\item 3 tranches d'ananas en dés
\item Une pomme
\item 10 amandes coupées en tranches
\item 1 cuillère de kirsch délayée dans une cuillère de gelé de groseille
\item Un jus de citron
}
Couvrir de crème fouettée et saupoudrer d'amande en tranche.\\
\\

\end{minipage}

\begin{minipage}[c]{\textwidth}
\recette{Beignet au pomme}
\ingredients{
\item 500g de farine
\item 3 oeufs
\item 100g de sucre (= 4 cuillères à soupe)
\item Une pincée de sel
\item Du lait
\item Une cuillère à soupe de levure
}
Incorporer les jaunes d'oeufs, le lait, le sel, la levure et le sucre à la farine jusqu'a obtenir une pate lisse. Ajouter légèrement les blancs en neige. Jeter sur la pate les rondelles de pommes au moment de mettre les beignets dans la friture bouillante.\\
Retourner avec une fourchette. Servir avec du sucre blanc.\\
\\

\end{minipage}

\begin{minipage}[c]{\textwidth}
\recette{Scones}
\ingredients{
\item 85 gr de margarine ou de beurre
\item 225gr de farine fermentante
\item 1 oeuf entier
\item 1 pincé de sel
\item Un peu de sucre
\item Raisins a volonté
}
Mélanger avec un peu de lait pour avoir une pâte a roulé puis couper en petite forme.\\
Cuisson a four chaud environ 20 min.\\
\\

\end{minipage}

\begin{minipage}[c]{\textwidth}
\recette{Confiture pomme au gingembre}
Peler et couper en quartier des pommes. Vider les et couper les en tranches assez épaisses. Ajouter pour un litre de pomme 3/4 de litre de sucre brun, 5 litres de jus, et des écorces râpées de 4 citrons, 250g de racines de gingembre ou de gingembre confit.\\
Laisser reposer 24h dans un bol. Bouillir jusqu'à ce que les pommes soit claires et aient une riche couleur d'ambre.\\
\\

\end{minipage}

\begin{minipage}[c]{\textwidth}
\recette{Tranches d pommes et gingembres}
Couper en petits morceaux 4 kilo de pommes sucrées non pelées. Ajouter 2 kg de sucre et 250g de gingembre confit. Laisser reposer 24h. Ajouter  4 citrons coupées en morceaux sans les pépins. Cuire lentement 3h. Mettre en bocal + parafins.\\
\\

\end{minipage}

\begin{minipage}[c]{\textwidth}
\recette{Massepain }
\otor{Tilou}
\ingredients{
\item 250g de sucre farine
\item 250g d'amande moulue
\item 2 cuillère à soupe d'eau
\item 1 cuillère à soupe d'eau de fleur d'oranger
\item 1 pincé de sel
}
Un blanc d'œuf en neige\\
Au bain marie\\
Versez le sucre petit à petit à l'eau (par cuillerée)\\
Jusqu’a faire ?\\
Versez petit à petit la poudre d'amande et entre temps pour détendre, une cuillère de blanc d'œuf battue\\
\\

\end{minipage}

\begin{minipage}[c]{\textwidth}
\recette{Soufflé au grand marnier}
Bouillir doucement 1/2 litre de lait et 150g de sucre\\
Dans une autre casserole, fondre 150g de beurre\\
Versez 40g de maïzena\\
Mélanger\\
Y mettre petit à petit le lait sucré pour faire une crème épaisse\\
Hors du feu\\
Versez un grand verre de liqueur grand Marnier, 5 jaunes d'œufs\\
Bien travailler\\
Ajouter 8 blancs en neige\\
Cuire à four moyen, préalablement chauffé, 16 à 18 minutes\\
\\

\end{minipage}

\begin{minipage}[c]{\textwidth}
\recette{Gauffre chaude }
\otor{Bonne Maman}
Un œufs + 1 cuillères de sucre + 1.5 cuillère de farine\\
à multiplier autant de fois qu'on veut\\
+ Un quart de beurre fondu? + Un peu de cannelle\\
Blanc en neige\\
Pour gauffres croquante mettre un peu d'eau\\
Cuire tout de suite\\
\\

\end{minipage}

\begin{minipage}[c]{\textwidth}
\recette{Tarte aux prunes }
\otor{Robert Ville}
200g de farine fermanté\\
80g de beurre\\
Mélangée en fontaine 2 œufs puis 10g de sucre\\
Recouvrir de prune\\
Saupoudrez le mélange de cannelle plus de sucre farine\\
Cuire le tout\\
\\

\end{minipage}

\begin{minipage}[c]{\textwidth}
\recette{Gâteau de famille au petit beurre}
Travailler ensemble 60g de beurre + 1 cuillère à café de sucre semoule, et un jaune d'œuf\\
Battre le blanc en neige ferme et mélanger légèrement à la pate\\
Trempez légèrement 24 petits beurres dans du café fort tiède ou froid\\
Étende une couche de biscuit puis une couche de crème alternativement\\
Recouvrir le gâteau terminé de chocolat râpé, ne pas cuire, manger le lendemain\\
\\

\end{minipage}

\begin{minipage}[c]{\textwidth}
\recette{Madeleine}
\ingredients{
\item 200g de farine
\item 225g de sucre fin
\item 2 paquets de sucre vanillé
\item 250g de beurre ou de margarine
\item 4 œufs
}
Faire fondre doucement le beurre (sans le laisser cuire) et refroidir\\
Dans un grand bol mélanger au fouet les jaune d'œufs et le sucre jusqu'au ce que ça mousse\\
Ajouter le beur fondu puis la farine petit à petit, une cuillère à café de jus de citron, et les zeste râpé fins\\
Battre les blancs en neige ferme et ajouter à la pâte en les soulevant délicatement\\
Mettre au frigo 40 min\\
Beurrer au pinceau les moules avec du beurre fondu, fariné, remplir les moule à 2/3 de pate\\
Cuisson 10 à 20 minute à four moyennement chaud, démoulé sur grille.\\
\\

\end{minipage}

\begin{minipage}[c]{\textwidth}
\recette{Tarte aux raisins}
Pate sablé avec 250g de farine tamisé, 100g de sucre, une pincé de sel\\
Ajouter un à un 2 œufs entiers\\
Puis 70g de beurre en petit morceau\\
Mélanger rapidement\\
Laisser reposer une heure\\
Abaisser au rouleau\\
Mettre dans le moule à tarte bien beurrer\\
Remplir de raisin frais\\
Cuire à four chaud 15 min \\
Sucrer à la sortie du four\\
On peut aussi napper d'un appareil à meringue (blanc + sucre battu) et dorer au four\\
\\

\end{minipage}

\begin{minipage}[c]{\textwidth}
\recette{Tarte frangipane}
Mélanger 150g d'amande douce, 100g d'amande amer avec un peu de lait\\
Ajouter 500g de sucre en poudre et 1/2 l de lait\\
Dans un bol mélanger à la spatule 250g de beurre\\
250g de sucre puis 5 jaune d'œuf entier un par un et 150g de farine\\
Mélanger les 2 pates\\
Cuire dans une platine à tarte à four chaud\\
\\

\end{minipage}

\begin{minipage}[c]{\textwidth}
\recette{Gâteau marrons noël (excellent !!)}
1 kilo 250 de pate de marrons glacé (ou de crème de marron un peu séchée à four très doux)\\
Et des marrons entier moulus\\
Incorporez 100g de beurre ramolli et une petite cuillerée de kirsch ou de cognac\\
Bien mélanger à la spatule en bois, laisser reposer au frais\\
Ensuite dans une terrine, versez et mélangez, 250g d'amande moulue, 150g de sucre en poudre, 150g de beurre, 2 jaune d'œuf, 3 cuillère à soupe d'extrait de café\\
Tapissé un moule à charlotte de 18cm de diamètre d'une mousseline (étamine à frittes) qui doit déborder largement\\
Tasser dans le fond la moitié de la pâte de marron au-dessus la pâte d'amande et couvrir avec la pâte de marron\\
Rabattre la mousseline sur le dessus du gâteau\\
Y poser une assiette avec un poids dessus, mettre au frigo une nuit, démouler le lendemain en tirant doucement l'étame\\
Glaçage:\\
    Faire fondre à feu doux 250g de chocolat, 2 cuillère à café d'eau, 2 cuillère à café d'extrait de café\\
    Remuer à la spatule jusqu'à obtenir une pate lisse\\
    Ajouter 50g de beurre,\\
    Mélangez\\
Trempez une spatule en métal dans de l'eau bouillante et l'utiliser pour lisser la crème au chocolat sur le gâteau et les coté\\
Garnir avec des demi-noix, des cerises confite ou des feuilles de houx\\
\\

\end{minipage}

\begin{minipage}[c]{\textwidth}
\recette{Gâteau à la noix de Cosette }
Ramollir 100g de beurre\\
Incorporer 200g de sucre\\
Battre a fouet jusqu'à ce que ça devienne blanc et mousseux\\
Incorporer peu à peu 200g de poudre de noix puis 4 jaune d'œuf un à un\\
Battre le balcon en neige très ferme\\
Les incorporer délicatement au mélange\\
Beurrer un moule à ? De 20 à 23 cm de diamètre\\
Cuire à four chaud 30 à 35 minute (jusqu'à ce qu'une aiguille piquée au milieu ressorte sèche, éventuellement couvrir d'un papier d'argent en fin de cuisson pour éviter de bruler)\\
Laisser refroidir dans le four éteint\\
Peut-être décoré de crème chantilly et de noix mais c'est déjà assez nourrissant\\
\\

\end{minipage}

\begin{minipage}[c]{\textwidth}
\recette{Spéculoos }
\otor{Catherine}
\ingredients{
\item 200g de farine
\item 175g de cassonade
\item 125g de beurre
\item 1/2 cuillère à café de cannelle
\item 1/2 cuillère à café de levure en poudre
\item 1/2 cuillère à café de bicarbonate de soude
\item 1 œuf
}
Tout mélanger\\
Laisser reposer une nuit au frais\\
Aplatir au rouleau,\\
Mettre dans le moule,\\
Cuisson 15 min à feu moyen\\
nb: les spéculoos sont mou à la sortie du four mais ils durcissent en se refroidissant\\
\\

\end{minipage}

\begin{minipage}[c]{\textwidth}
\recette{Crème anglaise }
\otor{Tilou}
Battre au fouet 2 jaune + 2 cuillère à soupe de sucre jusqu'à ce que ça blanchisse\\
Ajouter une cuillère à soupe rasé de farine (20g)\\
Délayer dans un peu de lait\\
Versez un grand verre de lait chaud (250ml) en tournant\\
Faire bouillir sur le feu,\\
Parfumez avec de la vanille, du cacao ou du café\\
Laisser refroidir en tournant de temps en temps pour éviter la peau\\
\\

\end{minipage}

\begin{minipage}[c]{\textwidth}
\recette{Gâteau sec }
\otor{Tilou}
\ingredients{
\item 3oeuf
\item 150g de sucre
\item 3 à 4 cuillère à soupe d'eau
\item 1 cuillère à soupe de jus de citron
\item 100g de farine
\item 100 de maïzena
\item 1 petit paquet de baking
}
Mélanger le jaune le sucre l'eau et le citron\\
Incorporer petit à petit la farine la maïzena le baking en alternant\\
Ajouter les blancs en neige\\
On peut ajouter avant les blancs 75g de beurre fondu et tiède\\
Cuisson 20 à 30 in à feu doux\\
\\

\end{minipage}

\begin{minipage}[c]{\textwidth}
\recette{Rombosses}
\ingredients{
\item 1 tasse de farine
\item 1/2 cuillère à café de sel
\item 1/3 tasse d'eau ou lait
\item 2 cuillères à café de baking
\item 2 cuillères à soupe de graisse
\item 4 pommes
\item 1/2 tasse de sucre
}
Faire une pâte molle et l'étendre au rouleau, la couper en forme carré, y mettre les pommes pelées et vider.\\
Remplir avec le sucre et la cannelle.\\
Mouiller les bords et en recouvrir la pomme. Percer à la fourchette pour laisser sortir la vapeur.\\
Cuire au four ou à la vapeur, la pomme doit être tendre. Servir avec du sucre et de la crème ou du sucre au citron. \\
\\

\end{minipage}

\begin{minipage}[c]{\textwidth}
\recette{Betty brune}
\ingredients{
\item 1 tasse de mie de pain
\item 8 pommes tranchées
\item 1 tasse de sucre
\item 1/2 tasse eau froide
}
Beurrer un plat allant au four et y mettre 1 couche de mie de pain, une couche de pomme, cannelle, sucre et beurre. Répéter jusqu'à ce que le plat soit rempli. Piquer avec 1 couteau pointu  et y verser l'eau et le sucre couverts du sirop. Faire cuire au bain marie 45 min. servir chaud avec de la crème.\\
\\

\end{minipage}

\begin{minipage}[c]{\textwidth}
\recette{Tarte aux noix }
1\degrees faire une pâte sablée en travaillant du bout des doigts et rapidement. Laisser reposer 1h.\\
2\degrees rouler la pâte au rouleau ou l'abaisser à la main dans un moule d'un diamètre de 24 cm.\\
3\degrees melangez environ 250g de crème très fraiche, 100g de sucre en poudre, 100 g de noix moulues.\\
4\degrees garnir la pâte de ce mélange et cuire à four moyen pendant 30 min (surveiller)\\
5\degrees refroidir la tarte sur une grille\\
6\degrees glacer avec 1 mélange de 100g de sucre glace (=sucre farine) et 2 cuillères à soupe de kirsch. \\
Etendre cette pâte coulante sur la tarte à la spatule. Laissez sécher quelques heures.\\
\\

\end{minipage}

\begin{minipage}[c]{\textwidth}
\recette{Pommes au gingembre}
Couper les pommes en quartiers et tranches épaisses.  3/4 du poids de sucre brun clair par kg de pommes, 1 zeste de citron râpé, 100 g de gingembre confit par kg de pommes, 1 noix. Couper en morceaux, reposé 1 jour, bouillir jusqu'à ce que la couleur ambre jaune très clair.\\
\\

\end{minipage}

\begin{minipage}[c]{\textwidth}
\recette{Régal aux pommes (conserves)}
\ingredients{
\item 7 litres de pommes surgelées en cube
\item 1/2 litres de noix ou noix de pékan 
\item 3 litres de sucre
\item 2 oranges (écorce râpées et jus)
}
Mélanger et laisser reposer 1 nuit pommes-sucre-orange et raisins.\\
Cuire lentement 45 min en remuant  et à couvert jusqu'à ce que les peaux soient absorbés. Ajouter les noix 5 min avant d'enlever du feu. S’emploie avec les tartes, la viande, les beignets ou à conserver dans des bocaux.\\
\\

\end{minipage}

\begin{minipage}[c]{\textwidth}
\recette{Gâteau fruits confits}
Battre 2 œufs et 2 jaunes. \\
Ajouter 3 tasses de sucre semoule. Battre. \\
Ajouter 1 petite tasse de Corinthe (lavées)\\
Ajouter 1 petite tasse de raisins blancs (sultans)\\
Ajouter 50g de fruits confits. Coupés au rhum.\\
Ajouter écorces d'oranges confites\\
Ajouter 260 g de beurre ramolli\\
Ajouter 1 pincée de sel\\
Ajouter par cuillerées 4 tasses farine tamisée et 1 paquet de baking.\\
Ajouter des blancs d'œufs en neige. \\
Laisser reposer la  pâte 4h ou plus. Faire cuire 50 min dans une forme bien beurrée et farinée, remplir au 3/4.\\
\\

\end{minipage}



\backmatter
\pagestyle{empty}


\ifodd\thepage\hbox{}\newpage\else\fi%si page paire ou impaire
\includepdf{images/4ecouv.pdf}


\end{document}