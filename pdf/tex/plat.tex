\begin{minipage}[c]{\textwidth}
\recette{Moules}
Laver et gratter les moules.\\
Les plonger dans de l’eau salée. Egoutter \\
Faire revenir un oignon coupé dans du beurre. \\
$\nicefrac{1}{2}$ céleri + persil. Faire revenir dans le beurre sans roussir. \\
Mijoter les moules. \\
Cuisson : environ 10 minutes\\
\\

\end{minipage}

\begin{minipage}[c]{\textwidth}
\recette{Soufflé au homard }
\ingredients[(6 personnes)]{
\item Bisque de homard : $\nicefrac{1}{2}$ à $\nicefrac{3}{4}$ de boite
\item 4 pommes
\item Gruyère râpé : 80 gr
\item 4 blancs d’œufs
\item Beurre
}
Verser $\nicefrac{1}{2}$ à $\nicefrac{3}{4}$ d’une boite de bisque de homard sans diluer dans une casserole de 1L, avec $\nicefrac{1}{4}$ de verre d’eau chaude. Chauffer à feu doux en mélangeant à la spatule. \\
Ecarter du feu. Ajouter les 4 pommes en mélangeant vite, puis ajouter le gruyère râpé et du poivre et du sel. \\
Battre très fermes 4 blancs d’œufs. Mélanger 2 cuillères à soupe de blanc battu à la préparation avec la cuillère puis le reste délicatement à la fourchette. \\
Beurrer le moule. Y verser le mélange aux $\nicefrac{3}{4}$ de la hauteur. \\
Cuire à feu moyen 15 à 20 min. \\
\\

\end{minipage}

\begin{minipage}[c]{\textwidth}
\recette{Foie}
Passer les tranches de foie dans de la farine avant de les faire cuire à la poêle. \\
\\

\end{minipage}

\begin{minipage}[c]{\textwidth}
\recette{Cassolette au homard}
Verser 1 boite de bisque de homard « Liebig » dans une casserole. Chauffer doucement en mélangeant. \\
Prélever quelques cuillérées de bisque éclaircies à l’eau chaude (un peu).\\
Ajouter 2 œufs frais et mélanger doucement au fouet sans faire mousser. \\
Incorporer le reste de la bisque. \\
Verser dans des ramequins remplis aux $\nicefrac{3}{4}$. Les poser dans un plat au four avec de l’eau chaude jusqu’à mi hauteur des ramequins. Cuire à four chaud pendant 25 minutes. \\
\\

\end{minipage}

\begin{minipage}[c]{\textwidth}
\recette{Spaghettis à l’italienne}
\ingredients{
\item Spaghettis
\item Lait
\item Levure
\item 2 œufs
\item Echalote
\item Jambon ou langue 
\item Sauce tomate
}
Cuire les spaghettis dans du lait (1/2 litre pour une poignée).\\
Ajouter hors du feu une cuillère de levure, 2 œufs battus et l’échalote hachée.\\
Hacher finement du jambon ou de la langue. Mélanger.\\
Cuire une heure au bain marie dans un moule à cheminée beurré. \\
Arroser de sauce tomate après cuisson. \\
\\

\end{minipage}

\begin{minipage}[c]{\textwidth}
\recette{Les œufs à la milanaise}
Beurrer le fond d’un plat à gratin. Y disposer de fines tranches de pain. \\
Recouvrir d’une couche de fines lamelles de gruyères, d’une couche de langue fumée hachée, d’une couche de sauce tomate. Casser dessus (sans crever le jaune) 3 ou 4 œufs. \\
Assaisonner. \\
Cuire à four doux pendant 10 minutes\\
\\

\end{minipage}

\begin{minipage}[c]{\textwidth}
\recette{Pontis auvergnat}
Dans un peu de beurre, faire remuer 3 rog lard fumé coupé en dés. \\
Ajouter un oignon moyen et une poignée de fines herbes hachées.\\
Cuire en mijotant avec un peu d’eau, du poivre et du sel. \\
Délayer deux tasses à thé de farine dans un bol de lait, ajouter 2 ou 3 jaunes d’œufs et les blancs battus en neige.\\
Mélanger le tout, puis mettre à cuire $\nicefrac{3}{4}$ d’heure dans un moule à soufflé grassement beurré. \\
\\

\end{minipage}

\begin{minipage}[c]{\textwidth}
\recette{Gratin dauphinois}
\ingredients{
\item 1 kg de pommes de terre
\item 2 œufs
\item 125 g de gruyère râpé
\item 75g de beurre
\item $\nicefrac{1}{2}$ litre de lait
\item Sel, poivre, muscade
}
Eplucher les pommes de terre et les couper en rondelles fines.\\
Battre les œufs en omelette et y mélanger le lait, sel, poivre et muscade. \\
Beurrer un plat allant au four, y placer une couche de pomme de terre. Arroser avec le mélange lait/œufs puis saupoudrer de râpé. Recommencer jusqu’à remplir le plat. \\
Arroser de beurre fondu sur le dessus du plat recouvert du reste de râpé. \\
Cuire à four chaud 60 à 80 min.\\
A manger comme entrée ou avec un plat de viande.\\
\\

\end{minipage}

\begin{minipage}[c]{\textwidth}
\recette{Petits pois à la française}
Faire revenir 4 tout petits oignons dans du beurre.\\
Ajouter quelques feuilles de laitue dans le fond et un peu d’eau si nécessaire.\\
1 bouquet de persil lié.\\
Verser les pois et 2 sucres. Laisser cuire à feu très doux une petite heure.\\
Retirer la laitue avant de servir.\\
\\

\end{minipage}

\begin{minipage}[c]{\textwidth}
\recette{Pate à frire très légère}
\ingredients{
\item 125 g de farine
\item 2/3 de verre d’eau, bière ou vin blanc
\item 1 cuillère d’huile ou de beurre fondu
\item 2 blancs d’œufs
}
Délayer la farine avec le liquide choisit plus une pincée de sel. Ajouter du beurre ou de l’huile quand c’est bien délayé.\\
Reposer une heure. \\
Au moment d’utiliser la pâte, y ajouter les deux blancs. Battre en neige ferme. \\
\\

\end{minipage}

\begin{minipage}[c]{\textwidth}
\recette{Beignets de quenelles de volailles}
\ingredients{
\item 1/2 boite de quenelles de +/- 35 g chacune
\item Pâte à frire très légère (recette précédente)
}
Égoutter les quenelles, puis les faire macérer une heure dans un peu de madère. Egoutter à nouveau. \\
Frire à l’huile après les avoir plongé une à une dans la pâte à beignet. \\
Egoutter et décorer de persil frit.\\
\\

\end{minipage}

\begin{minipage}[c]{\textwidth}
\recette{Cuisson du chou fleur}
Le mettre dan l'eau froide, portez à ébullition. Quand elle bout, jetez cette eau et la remplacer par de l'eau chaude.\\
(Ceci supprime la mauvaise odeur, idem pour les choux de Bruxelles)\\
\\

\end{minipage}

\begin{minipage}[c]{\textwidth}
\recette{Médaillon de veau}
Médaillons d’environ 4 cm d’épaisseur cerclés d’une bande de lard. \\
Faire fondre dans une cocotte un bon morceau de beurre, y mettre les médaillons et cuire environ 30 min (cocotte fermé) après les avoir doré des deux cotés.\\
Laver 250 g de champignons et jeter dans l’eau bouillante légèrement vinaigrée.  Faire bouillir 5 minutes, puis les égoutter dans un torchon. \\
Mettre les champignons dans la cocotte à coté des médaillons 5 minutes.\\
Retirer le tout, mettre dans la cocotte un bol de crème sur feu très doux sans faire bouillir mais lentement chauffé. \\
Y ajouter la viande et les champignons et servir.\\
\\

\end{minipage}

\begin{minipage}[c]{\textwidth}
\recette{Soufflé de jambon}
\ingredients{
\item 2 œufs 
\item 2 cuillères à soupe de farine
\item $\nicefrac{1}{2}$ L de lait
\item 2 cuillères de porto blanc ou Vermouth blanc
\item Beurre
\item Jambon en dés (250 gr)
\item Gruyère râpé
\item Sauce Mormay
}
Pâte : délayer deux jaunes d’œufs dans deux cuillères à soupe de farine tamisée avec \\
$\nicefrac{1}{2}$ L de lait bouilli et refroidi. Ajouter en battant 2 cuillères de porto blanc ou de vermouth blanc, du sel et du poivre puis les 2 blancs battus en neige ferme.\\
Beurrer un moule. Au fond, y placer une fine couche de jambon coupé en petits dés. Saupoudrer de gruyère râpé, puis verser une couche de pâte. Continuer en alternant les trois couches et terminer par la pâte.\\
Cuire au bain marie environ 1 heure. \\
Démouler sur un plat chaud et napper d’une sauce Mormay (pour sa confection, voir recette de la tarte aux asperges-Béchamel au lait et fromage râpe)\\
\\

\end{minipage}

\begin{minipage}[c]{\textwidth}
\recette{Timbale au jambon}
Cuire à l’eau salée 250g de coquillettes. Les égoutter dès qu’elles sont souples sous le doigt.\\
Rincer à l’eau froide et de nouveau égoutter. \\
Hacher assez gros 250 g de jambon maigre. Mélanger aux coquillettes avec 2 œufs battus et 125g de Gruyère râpé. \\
Beurrer largement le moule à cheminée ou à charlotte, et y verser la préparation. \\
Cuire au four ou au bain marie 30 minutes, puis laisser reposer dans le moule au bain marie. Après 10 minutes, démouler sur un plat préchauffé. \\
Verser dessus une sauce Aurore (Béchamel au lait et sauce tomate en quantité égale et chaude.)\\
\\

\end{minipage}

\begin{minipage}[c]{\textwidth}
\recette{Mousse de Jambon Florentine}
\ingredients{
\item Beurre
\item 3 cuillères de farine
\item 1,5 verre de lait
\item 1 cuillère de concentré de tomates
\item 300 gr de jambon maigre
\item 1-2 cuillère de crème épaisse
\item 2 œufs
\item Poivre
\item Epinards au beurre ou à la crème
}
Dans une casserole, ajouter 3 cuillères de beurre et de la farine en même quantité. Délayer à feu doux avec 1,5 verre de lait froid. \\
Laisser épaissir en tournant, puis ajouter une cuillère de concentré de tomates. \\
Cuire à l’étouffé environ 15 minutes avant d’ajouter le jambon maigre haché finement. \\
Ajouter rapidement  de 2 à 4 cuillères de beurre et 1 à 2 cuillères de crème épaisse. Réchauffer puis incorporer hors du feu les deux jaunes puis les deux blancs en neige ferme. Poivrer, puis verser dans un moule droit bien beurré. Cuire 15 min à four assez chaud. Démouler et entourer d’épinards au beurre ou à la crème. \\
On peut cuire la mousse dans une timbale en porcelaine en y ajoutant un blanc d’œuf en neige. Servir alors les épinards à part.\\
\\

\end{minipage}

\begin{minipage}[c]{\textwidth}
\recette{Tarte aux asperges }
\ingredients{
\item Pâte : 
\item 200 gr de farine
\item 150 gr de beurre
\item 1 petit verre d’eau salée.
}
Mélanger rapidement le tout, puis laisser reposer une heure. \\
Etaler dans un moule (29 cm). \\
Piquer, et cuire à four chaud pendant 20 min. \\
Cuire les asperges pelées dans l’eau légèrement salée de 20 à 30 min.\\
Déposer sur la pâte cuite les asperges cuites et coupées. \\
Recouvrir de sauce Mormay (voir ci dessous).\\
Glisser au four chaud quelques minutes pour gratiner (se réchauffe facilement si il en reste).\\
Sauce mormay :\\
Dans 30 gr de beurre fondu, jeter 30 gr de farine et mélanger. \\
Ajouter petit à petit 125 gr de crème ou lait condensé non sucré et 50 gr de fromage râpé. \\
En faire une crème épaisse. \\
\\

\end{minipage}

\begin{minipage}[c]{\textwidth}
\recette{Porc Maréchal}
(convient aussi pour un rôti de veau)\\
Cuire à four chaud un filet de porc de 1 kg ou un rôti un peu allongé, avec une cuillère de beurre pendant une heure (pas plus).\\
Laver 500 gr de champignons et les cuire au beurre puis les hacher avec 250 gr de jambon maigre. \\
Faire une béchamel épaisse : à 2 cuillères de beurre fondu, ajouter 5 cuillères de farine et réchauffer, tout en y versant progressivement $\nicefrac{1}{2}$ litre de lait. \\
Mélanger le haché dans cette sauce. Saler, poivrer et ajouter une pointe de muscade. Découper le rôti cuit en tranches pas trop fines et les recoller en utilisant le hachis entre chaque tranche.\\
Une fois le rôti remodelé, verser sur les cotés le reste du hachis et saupoudrer de 100g de fromage râpé (gruyère ou parmesan). \\
Faire gratiner au four chaud jusqu’à ce que le rôti et le fromage soient bien dorés. \\
\\

\end{minipage}

\begin{minipage}[c]{\textwidth}
\recette{Framboise surprise}
Moudre 125 gr d’amandes ou de noix mélangé à 100 g de sucre fin. Ajouter 3 jaunes d’œufs et bien mélanger. \\
Battre les 3 blancs en neige ferme et les ajouter au mélange très doucement. \\
Verser dans un plat qui va au four dans lequel on a mis 150 gr de framboises. Cuire 10 à 15 minutes à four doux.\\
Se déguste chaud ou froid !\\
\\

\end{minipage}

\begin{minipage}[c]{\textwidth}
\recette{Tajine de poulet au tomate et miel}
\ingredients{
\item 1,5 kg de poulet (cuisses par exemple)
\item 2 boites de tomates pelées
\item 2 oignons émincés
\item 4 cuillères à soupe de pignons
\item 2 cuillères à soupe de miel
\item Huile d’olive
\item Beurre
\item $\nicefrac{1}{2}$ cuillère à café de 4 épices
\item 2 cuillères à soupe de persil plat haché
}
Egoutter les tomates, les couper en deux. \\
Si c’est des grandes cuisses, les couper en deux aux jointures. Les faire dorer des 2 côtés dans l’huile et le beurre chauds. Retirer quand elles sont colorées et jeter l’excès de graisse. Faire revenir 2 oignons hachés saupoudrés de 4 épices.\\
Rajouter le poulet-tomates, le sel et le poivre, faire mijoter 45 minutes à feu doux en retournant les morceaux à mi cuisson.\\
Griller les pignons sans graisse. \\
Retirer le poulet, ajouter le miel et laisser mijoter 2 minutes. \\
Rajouter les morceaux de poulet et les pignons. Saupoudrer le tout de persil plat. \\
NB : c’est très bon et ca se surgèle facilement ! \\
\\

\end{minipage}

\begin{minipage}[c]{\textwidth}
\recette{Osso buco }
1. Mirepoix: \\
2 blancs poiveraux\\
2 carottes \\
2 oignons\\
2 branches de céleri\\
Tous ces ingrédients hachés fins. \\
100 gr de lard fumé haché\\
1 cuillère à soupe d’huile d’olive\\
Laurier et thym\\
Faire cuire à couvert très doucement jusqu’à ce que les légumes soient fondus, garder au chaud. \\
2. Jarret de veau : \\
6 tranches de 3 cuissons différentes. Les fariner, ajouter poivre et sel. Faire dorer à la poêle doucement dans 3 cuillères à soupe d’huile et 3 cuillères à soupe de beurre. \\
3. Ajouter la viande à la Mirepoix et $\nicefrac{1}{2}$ verre de vin blanc sec et faire mijoter jusqu’au moment où le liquide a réduit d’1/4. \\
Y ajouter 2 cuillères à soupe de purée tomate et une tasse de bouillon. \\
4. Faire mijoter 1h15. Passer le jus et faire réduire d’1/4. \\
Parsemer la viande de persil, d’ail et de zestes de citron hachés finement. \\
Napper les morceaux de veau de la sauce et servir bien chaud !\\
\\

\end{minipage}

\begin{minipage}[c]{\textwidth}
\recette{Champignons à la grecque}
\ingredients{
\item 1,5 verre d'eau
\item 1/2 verre d'huile
\item Le jus d’1 citron
\item 1 gousse d'ail brouillé
\item 6 grains de poivre
\item 6 grains de cerfeuil
\item 1 brin de fenouil
\item Du thym et du laurier
}
Versez le tout dans un casserole d'eau.\\
Mélanger et y faire bouillir les champignons petits ou coupés.\\
Cuire 8 minutes puis servir bien frais.\\
\\

\end{minipage}

\begin{minipage}[c]{\textwidth}
\recette{Tarte à la moelle}
1. Pâte : brisée. \\
250 gr (2x $\nicefrac{1}{2}$ tasse à thé) de farine\\
1 cuillère à café de sel\\
125 gr de beurre\\
Rouler le mélange avec les doigts jusqu’à obtenir une semoule. \\
Verser +- 1 verre d’eau mélangé pour avoir une pâte légère qu’on peut rouler facilement. Pétrir légèrement et reposer. \\
Beurrer le moule à tarte et fariner (+- 20 cm de diamètre et 2,5 cm de haut). Piquer, et bien appliquer sur les bords. \\
2. Garniture :\\
- tremper dans l’eau 50 gr de baguette ou petits pains (sandwich). \\
- macérer dans du vin rouge ordinaire +- 125 gr de raisin Corinthe toute une nuit\\
- oignons : 1 kg épluché et haché doré dans 3 cuillères à soupe d’huile. Ajouter 2 cuillères à   soupe de sucre\\
- moelle : évider 2 à 3 os pour obtenir +- 100-150 gr de moelle. Faire fondre à feu doux quelques minutes. \\
- exprimer l’eau du pain\\
- Ajouter les raisins égouttés\\
- Ajouter aux oignons\\
- Ajouter la moelle\\
Mélanger le tout à feu doux et mouler la pâte dans un moule enfariné de 26 cm +-. Piquer le fond et y verser le mélange. En garder un peu pour faire grésiller sur la tarte garnie. \\
Cuire à feu chaud 25 à 30 minutes. \\
\\

\end{minipage}

\begin{minipage}[c]{\textwidth}
\recette{Gâteau de crêpes}
Empiler en alternance 1 crêpe, 1 tranche de jambon, 1 crêpe, 1 tranche de jambon, etc. Jusqu’à ce que ça forme un gâteau. \\
Couler dessus une sauce au fromage faite en mélangeant dans un poêlon 35 gr de beurre, 1 cuillère à soupe de farine, du fromage, 1 jaune d’œuf et 1 bonne tasse de lait.\\
Découper comme un gâteau pour servir.\\
\\

\end{minipage}

\begin{minipage}[c]{\textwidth}
\recette{Poulet en gelée}
Dans une marmite : 200 gr de couenne de lard maigre (Facultatif : $\nicefrac{1}{2}$ pied de veau), 2-3 carottes, 2 oignons, 2 gousses d’ail, thym, laurier, persil, sel, poivre, 2 verres de vin blanc courant mais sec. Y poser le poulet et ajouter assez d’eau pour qu’il baigne.\\
Faire bouillir à petit feu +- 2h15. \\
Sortir le poulet et enlever toute la peau. Le désosser. \\
Remettre le bouillon au feu et laisser réduire pour avoir une gelée bien ferme. Passer dans la passoire fine. \\
Verser sur le poulet désossé et nu dans une terrine, les plus beaux morceaux en dessous. \\
En refroidissant, il y aura de la gelée sur le poulet. \\
Avant de servir, plonger la terrine 1 minute dans de l’eau chaude pour détacher le poulet et le renverser sur le plat de service. Y verser de la gelée (mais pas trop.) \\
Faire refroidir à l’air et servir avec une bonne salade. Mmm !\\
\\

\end{minipage}

\begin{minipage}[c]{\textwidth}
\recette{Courgettes aux tomates}
Faire fondre un peu de beurre dans un poêlon avec un de l’oignon finement haché (en quantité, l’équivalent d’une gousse d’ail). Y mettre les tomates dessus et les courgettes bien lavées et coupées en tranches. Ajouter in peu d’ail bien haché, du sel et du poivre. \\
Faire cuire 30 à 45 minutes.\\
Quand c’est cuit, mettre dans un plat à gratin et couler par dessus 2 œufs entiers battus.\\
Passer au four 5 minutes avant de servir. \\
\\

\end{minipage}

\begin{minipage}[c]{\textwidth}
\recette{Scaroles au fromage}
Cuire les scaroles à l'eau et les placez ensuite dans un pat allant au four.\\
Les recouvrir d'une sauce au fromage : une cuillère à café de farine, du beure, du fromage râpé, un jaune d'œuf.Laissez dorer au four.\\
\\

\end{minipage}

\begin{minipage}[c]{\textwidth}
\recette{Quiche de Lorraine}
Pâte :\\
\\
Faire un puits avec 250 gr de farine, y mettre 130 gr de beurre travaillé à la main, une pincée de sel et un petit verre de lait ou d’eau tiède. Mélanger le tout sans toutefois trop travailler la pâte. \\
La laisser reposer 2h au frais.\\
\\
Crème : \\
\\
Délayer 40 gr de farine avec un peu de crème (ou de lait).\\
Casser les 3 œufs et les travailler 5 minutes avec une cuillère en bois. Y ajouter le reste de crème et une pincée de sel. \\
Couper en petits carrés 100 gr de lard fumé.\\
Chauffer 5 gr de saindoux dans un poêlon. Y faire dorer le lard, l’égoutter et le mélanger avec la crème en remuant bien. \\
Etendre la pâte et la mouler dans une forme à tarte de 30 cm de diamètre préalablement beurrée.\\
Y verser la crème et cuire le tout à four chaud environ 30 minutes.\\
Servir chaud avec des épinards à la crème ou de la purée de pois.\\
\\
NB : Si la quiche se colore trop, la couvrir d’un papier beurré dès qu’elle à fait croûte.\\
\\

\end{minipage}

\begin{minipage}[c]{\textwidth}
\recette{Quenelles Rumsteak Américain }
Hacher un beefsteak d’un livre ?.\\
Y incorporer de la mie de pain rassis, 2 œufs entiers et du persil haché. \\
Bien mélanger.\\
Cuire au grill préalablement chauffé.\\
Sauter au beurre à la poêle. \\
\\

\end{minipage}

\begin{minipage}[c]{\textwidth}
\recette{Steak américain}
125 gr de viande hachée par personne.\\
2 jaunes d’œuf par 250 gr de viande.\\
Sel et poivre.\\
Ajouter une cuillère à café de moutarde délayée, de l’huile cf. mayonnaise (??), du vinaigre aromatisé, 1 filet de sauce anglaise, des petits cornichons hachés ou des câpres.\\
Servir avec des petits oignons sur le plat.\\
\\

\end{minipage}

\begin{minipage}[c]{\textwidth}
\recette{Pommes de terre à la mode de Prague }
Cuire à l’eau pendant 25 minutes des pommes de terre moyennes dans leurs pelures.\\
Cuire 6 œufs durs (10 min. dans l’eau bouillante).\\
Eplucher les pommes de terre et les œufs. \\
Couper les deux en rondelles d’1/2 cm. \\
Disposer une rangée de pomme de terre dans le fond du plat à gratin. Saler et poivrer. \\
Recouvrir d’une couche de crème fraiche, puis des œufs. Saler et poivrer. \\
Recouvrir du reste de pommes de terre. Saler.\\
Recouvrir du reste de crème (il en faut 150 gr pas trop épaisse.) saupoudrer de 50 gr de gruyères ou de parmesan râpé.\\
Mettre 10 minuntes à four chaud pour bien dorer. \\
Servir chaud !\\
\\

\end{minipage}

\begin{minipage}[c]{\textwidth}
\recette{Choufleur en timbale}
Cuire le chou à l'eau et le déposer par lit dans un moule beurré. Alternez avec un lit de viande haché jusqu'a épuisement.\\
Cuir au bain marie pendant 45 minutes. Servir avec une sauce tomate\\
\\

\end{minipage}

\begin{minipage}[c]{\textwidth}
\recette{Soufflé au fromage }
\ingredients[(6 personnes)]{
\item 250 gr de crème
\item 125 gr de gruyère râpé
\item 30 gr de beurre
\item 25 gr de parmesan râpé
\item 25 gr de fécule
\item 6 œufs frais
\item Muscade, sel et poivre. 
}
Mettre dans une casserole la crème, le beurre et la fécule.\\
Chauffer en mélangeant constamment jusqu'a obtenir une pâte lisse, assaisonner.\\
Eloigner la casserole du feu, y  ajouter les jaunes d’œufs et le fromage. Mélanger le tout.\\
Battre les blancs en neige bien ferme, les incorporer légèrement au mélange.\\
Verser le tout dans un plat creux légèrement beurré et mettre au four pas trop chaud pendant 25 à 30 min. \\
\\

\end{minipage}

\begin{minipage}[c]{\textwidth}
\recette{Pomme de terre soufflées}
Couper des pommes de terre en rondelles, les sécher et les étaler. Faire chauffer de l'huile pas trop chaude, y ajouter les rondelles une à une, cuire deux minutes à petit feu, enlever la bassine et laissez cuire 7 minutes.\\
Réchauffer le tout à gaz et ajoutez l'huile continuellement en tenant la bassine par les deux hanses. Lorsque les pommes de terre sont dorées et gonflent, retirer et saler.\\
\\

\end{minipage}

\begin{minipage}[c]{\textwidth}
\recette{Choux frisons}
Détacher les feuilles, blanchir à l'eau bouillante et puis les rafraichir. Hacher grossièrement. Mettre dans une casserole un peu de graisse, une demi verre d'eau et les choux. Assaisonner et ajouter deux ou trois cuillères de riz. Couvrir 20 min. Découvrir et saupoudrer de fromage. Gratiner au four quelques minutes\\
\\

\end{minipage}

\begin{minipage}[c]{\textwidth}
\recette{Semoule}
Faire bouillir un demi litre de lait avec 2 cuillères à soupe de sucre, ajouter 2 et demie cuillères à soupe de semoule en pluie et tournez avec le fouet jusqu'à ce que ca devienne très épais .\\
Goutez pour vérifier si c'est cuit, ajouter une poignée de raisin sec passé à l'eau chaude et un paquet de sucre vanillé. Retirez du feu et ajouter un jaune d'œuf. Incorporez un ou deux blanc battu en neige. Versez dans un plat et garnir de cassonade qui va fondre à la chaleur de la semoule.\\
\\

\end{minipage}

\begin{minipage}[c]{\textwidth}
\recette{Oignions à la monégasque}
Epluchez un litre d'oignions de même grosseur. Les mettre dans une casserole avec 1/3 de litre d'eau, deux verres à  porto de vinaigre blanc, 3 cuillères a soupe d'huile, 50g de sucre en poudre, 3 cuillères à soupe de purée de tomate, 1 petit bouquet de thym, du laurier, du persil, du poivre et du sel et 75 gramme de raisins secs. Faire bouillir et laisser cuir 1h30.\\
Servir froid dans un vasque en crystal\\
\\

\end{minipage}

\begin{minipage}[c]{\textwidth}
\recette{Poivron cru}
Couper en fines lamelles après avoir enlevé les pépins. Faire passer la nuit à la cave avec du sel et de l'huile (couvrir d'huile). Le lendemain ajouter de vinaigre.\\
\\

\end{minipage}

\begin{minipage}[c]{\textwidth}
\recette{Chanterelles}
Bien les laver et les presser dans un linge, les faire blanchir quelques instants à l'eau bouillante et salée. Les sauter au beurre avec du jus de citron à feu doux environ 20 minutes. Se sert mélangé à du jus de rôtis de veau et atour du rôtis avec une omelette ou des oeufs brouillés.\\
\\

\end{minipage}

\begin{minipage}[c]{\textwidth}
\recette{Cèpes farcies}
Bien laver. Séparer les tètes des queues que l'on réserve pour la farce. Hacher finement de l'oseil, du lard, des échalotes, de l'ail, de la ciboulette et du persil. Mélanger avec les queues. barder de lard le fond de la casserole. Disposer une bonne couche de champignons puis une couche de farce , recommencez jusqu'à épuisement.\\
Faire cuire 2-3 heures à petit feu.\\
Ajouter ensuite des pommes de terre bien rondes au dessus puis faire cuire encore deux heures.\\
\\

\end{minipage}

\begin{minipage}[c]{\textwidth}
\recette{Chateaubriand aux champignons}
Enlever les queues, éplucher les tête, couper en morceaux et tremper dans de l'eau acidulée. Mettre dan une petite casserole du jus de viande, un morceaux de beurre, du sel, du poivre et un petit verre de madère. Cuire à feu doux. \\
Préparer le bifsteak : Griller dans une poile noir, puis sauter au beurre. Pendant ce temps, faire cuir les champignons dans la sauce. Dresser le steak sur un plat garnir de champignons. Servir la sauce à part.\\
\\

\end{minipage}

\begin{minipage}[c]{\textwidth}
\recette{L’omelette mousseline}
Battre très ferme les blancs en neige de 6 œufs. Dans un bol battre les jaunes avec 100g de sucre puis incorporer les blancs, verser le tout dans un poêle beurrée.\\
Une fois cuite, passer au four pour dorer le second côté. Porter sur un plat chaud\\
Verser 3 cuillères de confiture en languette et replier vers le milieu les deux bords.\\
Arroser de kirsch enflammé\\
\\

\end{minipage}

\begin{minipage}[c]{\textwidth}
\recette{Aubergines niçoises}
Ouvrir six petites aubergines en deux en longueur, les faire revenir dans une poêle à l'huile pendant 10 min. Videz le milieu qu'il faut réserver pour la farce.\\
Mélangez le milieu avec 125g d’anchois ou de viande hachée, 30g de mie de pain (bouillie dans du lait jusqu'a ce qu'elle devienne une pate), une gousse d’ail finement haché, du sel et du poivre. Dresser les aubergines dans un plat à gratiné à côté de 500g de petites tomates ouvertes et vidées. Remplir les aubergines et les tomates de farce. Remettre le couvercle des aubergines/tomates et beurrer. Porter au four bien chaud qu'on baisse en enfournant. cuisson 1/2 heure. Servir brulant.\\
\\

\end{minipage}

\begin{minipage}[c]{\textwidth}
\recette{Bordure au fromage - ragout de crevettes ou champignons}
\ingredients{
\item 1/4 de litre de lait
\item 40g de beurre
\item 40g de farine
\item 100g de fromage rapé
\item 2 oeuf
\item Poivre et sel
}
Préparation : Faire fondre le beurre .Ajouter la farine et en tournant bien le lait tiède,\\
Cuire jusqu’à ce que le mélange devienne bien lisse. Ajouter les jaunes un à un, le fromage et le sel. Battre les blancs en neige ferme et ajouter légèrement au mélange.\\
Beurrer et saupoudrer de chapelure un moule à cheminer.\\
Cuire au bain marie une bonne demie heure\\
Démouler sur un plat chauffé et remplir le creux avec des crevette ou des champignons.\\
\\

\end{minipage}

\begin{minipage}[c]{\textwidth}
\recette{Tomates farcies à la forestière }
\ingredients[(6 personnes)]{
\item 6 à 9 assez grosse tomates
\item 750g de champignons
\item 2 oeufs
\item 60g de beurre
\item 3 oignons moyens
\item 1/2 cuillère à café de sel fin
\item Quelques tours de moulins à poivre
\item 1 cuillère à soupe persil haché et fines herbes hachées.
}
Préparation :\\
Eplucher les oignions et passer les au moulin à julienne. Les cuire a feu très doux dans 30g de beurre. Couper la partie sableuse des champignons, les laver et les passer au moulin à julienne. Ajouter la moitié du sel, 4 tours de moulins à poivre. Mélanger. Couvrez. Cuire à feu très vif 3 à 4 minutes. Retirer le couvercle pour laisser évaporer l'eau. Laver et essuyer les tomates. Couper en rondelle. les vider puis saler et poivrée chacune des rondelle dans le plat et mettre à four moyen 4 à 5 min.\\
farcissage :\\
égoutter les champignons, battre les oeufs en omelette, les ajouter aux champignons et aux herbes hachées. Sortir le plat du four et remplir les tomates avec le mélange à base de champignons. Couvrir avec les rondelles enlevées. Poser dessus une petite coquille de beure. Cuire à four moyen 20 à 30 minutes. Servir tel quel ou avec un roux blond :\\
30 g de beurre\\
30g de farine\\
et l'eau de cuisson des champignons.\\
\\

\end{minipage}

\begin{minipage}[c]{\textwidth}
\recette{Poulet au four à la tomate}
Couper le poulet en 4 morceaux. Enduire la casserole avec 10 gousses d'ail pillés, y mettre le poulet après avoir éventuellement ôté l'ail. Ajouter après 1/4 de cuisson une demie boite de tomate pelées, chauffer, épaissir la sauce avec de la farine.\\
\\

\end{minipage}

\begin{minipage}[c]{\textwidth}
\recette{Tarte au fromage }
\otor{Bonne Maman}
Mélange 1 :\\
    600g de fromage blanc\\
    une noix de beurre\\
    une cuillère a soupe de farine\\
    deux cuillère a soupe de sucre fin\\
    un oeuf entier\\
    une pincée de baking\\
    un zeste de citron \\
\\
Mélange 2 :\\
    200g de crème\\
    2 cuillère à soupe de sucre\\
    1 cuillère à soupe de farine\\
    un jaune d'oeuf\\
    1 ? farine fermantante\\
    Bien mélanger, y ajouter deux blancs en neige (le 2e reste du jaune mis dans la pâte)\\
\\
La pâte :\\
    125g de beurre, tourné en crème\\
    2 cuillère à soupe de sucre\\
    1 jaune d'oeuf\\
    200g de farine\\
    2 petites cuillères à café de farine fermantante\\
    un peu de lait si nécessaire\\
Rouler la pate et la mettre à la main dans le moule. Ajouter un peu de crème au fromage blanc (mélange 1)\\
Verser le mélange 2? sur la pâte et verser le reste de la crème par dessus. Cuir à four moyen.\\
\\

\end{minipage}

\begin{minipage}[c]{\textwidth}
\recette{Cèpe farcis limousin}
\ingredients{
\item Tête entière,
\item Oseille
\item Échalotes
\item Ail
\item Persil
\item Ciboulette
\item Lard
\item Hachez le tout
\item Foncez la casserole de bande de lard revenu au beurre
\item + d’une couche de champignons
\item Y ajouter une couche de farce (légumes hachés + un peu d'eau + un rien de farine pour lier, sel, poivre)
\item Cuire 2 à 3 heures à chaleur douce au four au bain marie
\item Ajouter des pommes de terre rondes, épluché salé et poivré
\item Cuire encore 2h à four doux
}

\end{minipage}

\begin{minipage}[c]{\textwidth}
\recette{Terrine de lièvre}
Pour 3 lièvres\\
Hachez un demi-livre de veau et un demi-livre de porc\\
Haché la viande de lièvre crue et 3 échalotes\\
Mélanger le tout en laissant de belles tranches fines dans le filet?\\
Tapissez la terrine de lard fumée\\
Tassez la farce\\
Mettre les tranches au milieu\\
Fait bouillir les os avec un os de veau, du thym du laurier et un clou de girofle\\
Laissez réduire jusqu'à ce que ça devienne gélatineux\\
En couvrir le pâté\\
Soudez avec un peu de patte faite avec un mélange farine et eau\\
\\

\end{minipage}

\begin{minipage}[c]{\textwidth}
\recette{Poulet à l'orange (ou perdreau)}
Colorez le poulet au beurre\\
Taillez en fine lanière l'écorce d'une petit orange jeté dans l'eau bouillante\\
Versez le jus dans un bol\\
Ajouter un verre de porto ou de curaçao\\
Découpez la viande en deux ou trois morceau\\
Versez dessus le mélange du bol dans la sauce\\
Bouillir\\
Ajouter un peu de fécule (1 càc) délayé dans du porto pour délayer la sauce\\
Garnier des quartiers d'orange accru\\
\\

\end{minipage}

\begin{minipage}[c]{\textwidth}
\recette{Compote anglaise}
éplucher les quartiers de pomme\\
Faire sirop épais 100g de sucre frotté? \\
Mélanger le jus obtenus en premier avec sirop au jus d'orange au fouet plus une cuillère à café de maïzena délayer à froid\\
Quartier dans sirop chaud\\
Rebouillir\\
En couvrir les quartiers\\
\\

\end{minipage}

\begin{minipage}[c]{\textwidth}
\recette{Cèleri rémoulade }
Deux boules de céleri rave (environ 1kg)\\
Éplucher, couper en tranche et râpé\\
2 cuillères à café foutée de mayonnaise à la moutard\\
Reposé une minute, ajouter 3 pincé de sel, du poivre et une pointe de curry\\
Verser huile à gros goulot détendue avec du jus de citron\\
Une cuillère à soupe de crème laissé macérer au moins une heure\\
\\

\end{minipage}

\begin{minipage}[c]{\textwidth}
\recette{Anguille au vert }
\otor{Tilou}
Pour 1 kg\\
Bien sécher les anguilles ?\\
Les raidir dans 50g de beurre très chaud (mais non brun) quelque minutes\\
Ajouter un forte poignée d'herbe hachée:\\
    Deux échalotes en premier\\
    Puis persil\\
    Sauge\\
    Oseille\\
    Cerfeuil\\
    Marjolaine\\
    Romarin\\
    Épinard et citronnelle\\
    Un jus de citron\\
Cuire 2 minutes\\
Ajouter un demi vin blanc ou moitié vin blanc moitié eau, jusqu'à couvrir\\
Laisser cuire 5 min\\
Lier avec une cuillère à café de fécule délayé dans 2 jaune d'œufs et un peu de jus de cuisson\\
Versez sur les anguilles et versez le tout dans une terrine\\
Laissez refroidir\\
\\

\end{minipage}

\begin{minipage}[c]{\textwidth}
\recette{Paupiette}
\ingredients{
\item Escalope de dinde ou de veau
\item Tranches de jambon
\item Éventuellement une tranche de fromage de hollande.
}
Rouler et ficeler le tout\\
Cuire à feu doux dans une cocotte environ une heure avec un morceau de carotte et oignions émincés, du beurre et éventuellement un verre de porto ou de vin blanc\\
Bien arroser\\
\\

\end{minipage}

\begin{minipage}[c]{\textwidth}
\recette{Brochette au porc }
Un filet de porc\\
Le panner entier dans de la farine puis dans des œufs battu puis dans de la chapelure Knorr (mélange spéciale) additionnée d'une bonne pincée de curcuma\\
Ensuite le couper en grosse tranches,\\
On frit les deux surfaces non pannée,\\
On bâti ensuite les brochettes\\
Et on les bâtie porc lard fumé poivron et oignions (passés à l'eau froide qu'on fait bouillir quelque minute puis on y fait cuire les poivrons 5 min)\\
Parsèment d'herbe de Provence et huiler,\\
Cuisson 20 min\\
Servir avec du riz\\
\\

\end{minipage}

\begin{minipage}[c]{\textwidth}
\recette{Salade au lard}
Cuire à l'eau froide 1 kg de pomme de terre (belle de Fontenay) en pelure ? Jusqu’à pouvoir les transpercer (environ 15 min)\\
Éplucher les, couper en grosse tranche dans un saladier\\
Dans une poêle :\\
3 cuillères à soupe d'huile d'arachide\\
Y faire fondre à feu doux 300g de lard coupé en lamelle\\
Mesclun :\\
    Mélange de salade :\\
        Romaine, frisée, mâche, roquette (amère)\\
    200g de lard maigre fumé, coupé en lardons mélanger au lard gras\\
Saler et poivrer les pommes de terre\\
Ajouter le mesclun, mélangé, ajouter le lard\\
Verser 2 cuillères de vinaigre de vin blanc à l'ancienne dans une poele chaude\\
Verser sur le saladier et servir\\
\\

\end{minipage}

\begin{minipage}[c]{\textwidth}
\recette{Salade d'hiver }
Verser dans une cocotte 4 cuillère à soupe d'huile d'olive, y faire blondir à feu doux 2 échalotes hachées finement\\
Couper 6 chicons en 4 en longueur\\
Bien laver 500g champignons de paris\\
Détacher en petit bouquet, un petit chou-fleur, les mettre dans un saladier couvert d'eau froide, d'un demi jus de citron, de thym et de laurier\\
Monter le feu\\
Ajouter 40 cl de vin blanc à sec (plus 2 verres à vin)\\
2 petits piments écher\\
Une pincé de sucre\\
Une cuillère à café de grains de coriandre\\
2 gousses d'ail (enlever le germe, coupé en 2)\\
Une cuillère à café de concentré de tomate\\
Ajouter le chou-fleur\\
Les chicons\\
Et les champignons\\
Faire bouillir jusqu'à ce que ça monte (environ 5 minutes)\\
Puis mélanger\\
Cuire 15 lins à gros bouillon\\
Verser dans le premier saladier avec le liquide.\\
Laisser refroidir jusqu'au lendemain\\
Ensuite, ajouter 4 cuillères d'huile d'olive\\
Servir avec du Persil\\
\\

\end{minipage}

\begin{minipage}[c]{\textwidth}
\recette{Civet de lièvre }
Mettre au four 10 à 15 min le lièvre enrobé de moutarde avec 1 carotte. 1 oignon coupé, du tin et du laurier. \\
Ajouter 1 petit verre de vin blanc quand le jus est brun. 1 verre de cognac qu'on flambe (??). \\
Avec la sauce faire 1 sauce liée à la farine avec crème ou lait.  Ajouter  1 cuillère à soupe de gelée de groseilles et un peu de jus de citron.\\
\\

\end{minipage}

\begin{minipage}[c]{\textwidth}
\recette{Hamburger }
\otor{Tilou}
\ingredients{
\item 150g de haché de bœuf par personnes
\item Une demie cuillère à café de paprika
\item 1/4 cuillère à café de thym
\item 1 gonion
\item 1/2 tasse à thé de bouillon
\item Une cuillère à soupe de ketchup
\item 1/2 cuillère à café de sauce anglaise
}
Faire 2 boulles de viande par personne\\
Y mélanger le paprika le thym l'oignion le sel poivre\\
Intercalé de fine tranches ?\\
Cuire à la  poêle dans du beurre + sauce et bouillon\\
Servir bien chaud\\
\\

\end{minipage}

\begin{minipage}[c]{\textwidth}
\recette{Coquille à la niçoise}
Cuire à l'eau bouillant salé 150g de spaghetti\\
Laver mais sans les éplucher 150g de champignon\\
elincer les tètes\\
 Faite les macérer 30 min dans du jus de citron\\
 Préparer un mayonnaise épaisse\\
 Y mélanger les spaghettis égouttés\\
 12 olives noir dénoyauté\\
 2 à 3 cuillères de pulpe de tomate (débarrassée de la peau, pépin et excès d'eau)\\
 Disposer le mélange dans des coquilles individuelles\\
Décoré de rondelle de piment rouge\\
Servir glacé\\
\\

\end{minipage}

\begin{minipage}[c]{\textwidth}
\recette{Poitrine de veau farcie }
\ingredients[(8 personnes)]{
\item 80g de pain à gonfler dans du lait chaud
\item 200g de lard fumé maigre en cure
\item 200g gorge fraiche de porc
\item 1 bouquet de persil (coupé les queue)
\item 2 gousse d'ail coupée en deux et enlever le germe
}
Moudre le tout\\
Ajouter à la farce une cuillère à café de parki, une cuillère à café de sel quelque tour de moulin à poivre et 3 œufs entiers\\
2 kilo de poitrine de veau ouverte en 2, salé et poivrer légèrement, tartinez de la farce\\
Rouler et ficeler le tout, sel poivre\\
Dans une cocotte, les faire dorer à feu vif dans 3 cuillère à soupe d'huile avant de les enlever du feu\\
Haché grossièrement 2 oignions et 2 gousse d'ail\\
Dorer les oignions dans la cocotte\\
Ajouter l'ail\\
20cl de porto\\
Du sel et du poivre\\
y remettre le veau et placer le tout au four à 180 degré pendant 2h15\\
Couper en tranche 1kg de champignons\\
Ajouter à la poitrine, 5 min à feu vif\\
Enlever le rôti\\
Ajouter 2 cuillère à soupe de crème au champignon,\\
Mélanger à gros bouillon à découvert\\
Enlever les files du rôti et découper en tranches\\
\\

\end{minipage}

\begin{minipage}[c]{\textwidth}
\recette{Escalope de veau à la crème}
4 fine escalopes\\
Saupoudrez-les d'un peu de sel et de poivre\\
Passez-les dans très peu de farine\\
Chauffer 2 cuillères de beurre et un peu d'huile (pour empêcher de noircir)\\
y cuire les deux escalopes 5 min de chaque coté\\
Garder au chaud dans le jus de cuisson faite bouillir 2 dcl de vin blanc sec\\
Laisser réduire de moitié\\
Ajouter hors du feu 2 cuillères de beure\\
Napper les deux échalotes de la sauce et recouvrez de 100g de crème chauffé au bain marie\\
Servir sans mélanger la sauce et la crème\\
\\

\end{minipage}

\begin{minipage}[c]{\textwidth}
\recette{Pâté au poivre vert}
Éplucher et haché 6 échalotes\\
Mélanger :\\
    500g de chair a saucier (haché veau/porc)\\
    500g de haché de veau\\
    500g d'escalope de dinde haché\\
    3 œufs entiers\\
    Une cuillère à soupe de poivre vert en grain\\
Saler et épicer, vérifier l'assaisonnement\\
Tasser dans une terrine\\
Faire cuire +- 1h30 à 200deg\\
Laisser refroidir\\
Recouvrer de poivre vert (pas trop) et une cuillère de jus contenus dans la boite, servez avec une salade \\
\\

\end{minipage}

\begin{minipage}[c]{\textwidth}
\recette{Porc aux pruneaux }
6 filets mignons de porc\\
Salé poivré\\
Cuit (doré) à feu vif pour saisir la viande dans 70 g de beurre, garder au chaud\\
500g de pruneau trempé dans du vin ou du porto la veille\\
Les cuire 1/2 dans le vin de trempage\\
Réduire sans les pruneaux\\
L’ajouter au jus de la viande\\
Faite réduire le tout à feu vif en tournant puis ajouter 200g de crème de épaisse (ou une petite boite de lait concentré) et une cuillère de gelé de groseille\\
Servir la viande entourée des pruneaux chauds\\
La sauce à part dans un theillert chauffée\\
\\

\end{minipage}

\begin{minipage}[c]{\textwidth}
\recette{Poulet à l'estragon}
farce :\\
    foie de poulet haché\\
    une tranche de pain rassi mouillé dans du lait chaud ? ?\\
    20g d'estragon frais haché\\
    muscade\\
    sel poivre\\
    mélanger le tout,\\
    fare revenir au beurre\\
    laisser refroidre\\
incorporer un jaune d'oeuf\\
remplisser le poulet de cette farce et suturer le\\
cuire au beurre dans une cocotte ou au four.\\
revenir jusqu'a obtenir un belle couleur\\
mouiller avec un peu d'eau\\
cuir au moins 30 min avec ou sans couvercle\\
Sauce :\\
    Ajouter au jus de la viande une forte cuillerée de farine en tournant\\
    puis 125g de crème (ou lait condensé en boite)\\
    et 10 feuilles d'estragon entière\\
    Donnez à la sauce un sel bouillon\\
    Servez bien chaud\\
\\

\end{minipage}

\begin{minipage}[c]{\textwidth}
\recette{aubergine farcie au fromage }
\otor{Mina}
\ingredients[(6 personnes)]{
\item 6 aubergines ronde de 200g
\item 500g de fromage de chèvre fars
\item 30g de parmesan
\item 2 œufs
\item 150g d'oignon
\item 1 cuillère à soupe d'huile d'olive
\item 2 pincé de muscade
\item Sel poivre
}
Couper un chapeau de 2 cm de haut aux aubergines\\
Les cuire 15 min à la vapeur égoutté\\
Farce :\\
    Mélanger le fromage de chèvre, le parmesan, les œufs et les épices\\
    Haché menu les oignions\\
    Et faire très légèrement dorer dans de l'huile d'olive\\
Laisser refroidir l’aubergine puis les creuser à la cuillère en laissant 2 cm de coque\\
Ecraser la pulpe et l'ajouter aux oignions dorer\\
Cuire 3 à 4 minutes\\
Ajouter à la préparation au fromage, la pulpe et les oignions ainsi que du persil haché\\
Farcir les aubergines du mélange\\
Couvrir des chapeaux\\
Verser le reste d'huile dans un plat à gratin pouvant juste contenir les aubergines et le y poser\\
Cuire 35 à 40 minutes dans le four préchauffé (180deg)\\
Servir chaud ou tiède\\
Nb: peut se préparer avec des courgettes\\
\\

\end{minipage}

