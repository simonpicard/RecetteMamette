\begin{minipage}[c]{\textwidth}
\recette{Potage au curry}
Couper en petits dés : 4 carottes, 2 blancs de poireaux. Emincer un oignon et une gousse d’ail. Cuire $\nicefrac{1}{2}$ heure à petit feu dans un fond d’eau avec 2 cubes de bouillon. Ajouter une cuillère à dessert de curry. \\
Passer au mixage de la soupe. \\
\\

\end{minipage}

\begin{minipage}[c]{\textwidth}
\recette{Mousse de saumon}
\ingredients{
\item Une boite de saumon au naturel 425-450 gr
\item Un sachet de gelée royale
\item Une petite boite de crème épaisse
\item Citron
\item Poivre
}
Diluer la gelée dans $\nicefrac{1}{2}$ L d’eau. Porter à ébullition puis retirer du feu. \\
Ajouter la crème, le saumon égoutté, $\nicefrac{1}{2}$ jus de citron et le poivre. Bien mixer le tout. \\
Mettre au frigo au moins 5h. \\
\\

\end{minipage}

\begin{minipage}[c]{\textwidth}
\recette{Mousse aux foies de volaille}
\ingredients{
\item +- 450 gr de foie
\item 1 sachet de gelée royale
\item 1 petite boite de crème épaisse
\item cognac ou porto
}
Rôtir le foie jusqu’à ce qu’il soit bien cuit. \\
Flamber au cognac ou mettre du porto, ajouter du sel et du poivre. \\
Continuer comme la recette précédente, sans le citron bien sûr. \\
\\

\end{minipage}

\begin{minipage}[c]{\textwidth}
\recette{Salade italienne (excellent !) }
\otor{Bonne Maman Picard}
Couper en petits morceaux : \\
6 grosses pommes de terre cuites\\
La même quantité de betteraves rouges cuites (ou un pot de betteraves au vinaigre)\\
1 dés de céleri rave cuit\\
1 oignon\\
1 pomme crue\\
$\nicefrac{1}{2}$ concombre\\
1 à 2 cuillères à soupe de petits pois (boite)\\
1 boite d’anchois à l’huile\\
Mélanger avec poivre et sel\\
1 à 2 cuillères à soupe de viande cuite\\
Mélanger avec mayonnaise épaisse\\
Décorer avec œuf dur haché, persil, betterave rouge hachée\\
\\

\end{minipage}

\begin{minipage}[c]{\textwidth}
\recette{Soupe à l'orge}
Cuire pendant +- 1 heure de l'orge dans de l'eau avec une gousse d'ail entière, du thym et du laurier. Ne pas saler.\\
Cuire 20 minutes dans de l'eau quelques pommes de terre et un cèleri coupé en morceaux. Saler.\\
Mélanger les deux.\\
\\

\end{minipage}

\begin{minipage}[c]{\textwidth}
\recette{Consommé fin }
\otor{Tilou}
Après les 2h30 de cuisson, ajouter 1 livre ? de bœuf haché et laisser le tout encore cuire 1h.\\
Enlever la viande avant de servir et l’utiliser comme les restes de viande.\\
\\

\end{minipage}

\begin{minipage}[c]{\textwidth}
\recette{Soupe à l’oignon (gruau) }
\otor{Tilou}
Faire revenir un gros oignon dans du beurre (non-bruni !). Le laisser suer à feu très doux. Couvrir d’eau. Ajouter un céleri, une poignée de gruau, du poivre et du sel.\\
Laisser le tout cuire environ 25 minutes. \\
Passer le potage avant de servir.\\
\\

\end{minipage}

\begin{minipage}[c]{\textwidth}
\recette{Soupe au cresson }
\otor{Tilou}
Jeter dans l’eau une botte de cresson, un céleri coupé, du poivre, du sel, du thym, du laurier et une gousse d’ail.\\
Ajouter une poignée de riz quand ça bout.\\
Après environ 25 minutes de cuisson, passer le tout.\\
Au moment de servir, ajouter du liebig et éventuellement un jaune d’œuf dans le fond de la soupière.\\
\\

\end{minipage}

\begin{minipage}[c]{\textwidth}
\recette{Fondues au fromage}
Même pâte que pour les croquettes aux crevettes mais sans les crevettes et en doublant la quantité de fromage (100 gr de gruyère râpé et 50 gr de parmesan râpé). \\
\\
NB : Toujours mettre à demi moins de parmesan par rapport aux autres fromages.\\
\\

\end{minipage}

\begin{minipage}[c]{\textwidth}
\recette{Bouillon }
\otor{Tilou}
Dans l’eau chaude mettre 2-3 carotte, 2-3 poireaux et 1 céleri (environ le tout par parts égales). \\
1,5 livre ? de bouillis au moins.\\
1 livre ? d’os à moelle si possible.\\
Du poivre, du sel, du thym, du laurier et de l’ail.\\
Faire brunir 1 oignon au four jusqu’à ce qu’il soit bien brun.\\
2h30 de cuisson douce.\\
Porter le tout à ébullition puis sur flamme frisonnante.\\
Ajouter des pâtes au moment de servir.\\
\\

\end{minipage}

\begin{minipage}[c]{\textwidth}
\recette{Croquettes aux crevettes}
100 gr de crevettes épluchées (ce qui correspond environ à 250 gr de crevettes non épluchées).  \\
Les faire suer dans un poêlon couvert à feu très doux. \\
Dans un autre poêlon, faire fondre 100 gr de beurre. Y ajouter 100 gr de farine et les trois quarts d’une pinte de lait chaud, du sel, du poivre et 50 gr de fromage. Bien mélanger le tout jusqu’à ce que la pâte se détache de la casserole (sinon elle casse dans la friture). \\
Hors du feu, incorporer un à un, 2 ou 3 jaunes d’œufs à la pâte puis les crevettes.  \\
Remettre un instant sur le feu. \\
Mettre la pâte sur une plaque huilée et mouler les croquettes à l’épaisseur voulue. \\
Laisser refroidir.\\
Passer les croquettes dans la farine, le blanc d’œuf et la chapelure.\\
Laisser reposer.\\
Cuire à la friture.\\
\\

\end{minipage}

\begin{minipage}[c]{\textwidth}
\recette{Bouchées aux crevettes}
Remplir les ridés de la même pâte que pour les croquettes aux crevettes et passer les au four 5 minutes. \\
\\

\end{minipage}

\begin{minipage}[c]{\textwidth}
\recette{Riz de veau aux champignons}
Etape 1: La viande\\
Mettre dégorger la viande dans de l’eau salée ($\nicefrac{1}{2}$ cuillère à soupe de sel).\\
Changer d’eau. \\
Mettre la viande à cuire avec un oignon, une feuille de thym et de laurier, 2 pincées de sel et de poivre.\\
Cuisson une grosse demi-heure.\\
Décortiquer la viande une fois cuite. \\
\\
Etape 2: La sauce (200 gr. de sauce aux champignons pour 350 gr. de riz de veau)\\
Bien laver les champignons à plusieurs eaux.\\
Couper les pieds des têtes, puis égoutter. Faire fondre un gros morceau de beurre jusqu’à ce que ça devienne mousseux. \\
Y faire sauter les champignons à vive allure jusqu’à absorption de tout le beurre. \\
Ajouter 2 cuillères à soupe de farine et $\nicefrac{1}{2}$ jus de citron. Les incorporer totalement. \\
Bien couvrir de lait. Cuire à vive allure 5 minutes en secouant la casserole de temps en temps puis cuire à feu doux 20 à 30 minutes.\\
\\
Etape 3: Les boulettes\\
Pour allonger on peut mettre des boulettes de veau.\\
Mélanger 100 gr de viande de veau haché, 1 jaune d’œuf, du sel et du poivre. \\
Ajouter une poignée de persil haché. Faire absorber du lait par une tranche de mie de pain     au feu avant de l'incorporer à la viande. \\
Former des petites boulettes de viande.\\
Les cuire 5 minutes dans du bouillon (ou un cube dans de l’eau bouillante).\\
\\
Ajouter le riz de veau en tranches et les boulettes à la sauce aux champis ! Bon appétit \\
\\

\end{minipage}

\begin{minipage}[c]{\textwidth}
\recette{Champignons à la grecque}
Dans une casserole versez :\\
un verre d'eau et demie, un demie verre d'huile, un jus de citron, une gousse d'ail brouillé, 6 grains de poivre, 6 grains de cerfeuil, un brins de fenouil, du thym et du laurier. Mélanger et y faire bouillir des petits champignons ou des champignons coupés.\\
Laisser cuire 8 min.\\
Servir bien frais.\\
\\

\end{minipage}

\begin{minipage}[c]{\textwidth}
\recette{Soupe à l'orge}
Faire cuire de l'orge dans de l'eau avec une gouse d'ail entière, du thym et du laurier. Ne pas saler. Faire cuire pendant 1h environ. Faire cuire 20 min dans de l'eau quelques pommes de terre et un cèleri coupé en morceaux, saler et ensuite les rajouter à la soupe.\\
\\

\end{minipage}

\begin{minipage}[c]{\textwidth}
\recette{Potage au pourpier}
Cuir du riz dans du bouillon, avec un cèleri coupé en morceaux. Passez et assaisonnez.\\
Y ajouter le pourpier avant de servir.\\
\\

\end{minipage}

\begin{minipage}[c]{\textwidth}
\recette{Pourpier en légume}
Nettoyer le pourpier divisé en feuilles, le cuire dans du beurre mousseux ,environ 1 litre pour deux personnes.\\
Cuir environ 10 minutes.\\
\\

\end{minipage}

\begin{minipage}[c]{\textwidth}
\recette{Concombre à la grec}
Hors d'œuvre à servir glacé.\\
Détaillez en quartiers deux concombres blancs. Mettre dans une cuisson bouillante de 4dc d'eau, 1dcl d'huile et deux jus de citron, un bouquet garni composé de cèleri, de fenouil, de persil, de thym et de laurier.\\
Ajouter 20 graines de cerfeuil, du poivre et du sel.\\
Cuire à vive allure 10 à 12 minutes.\\
Retirez le bouquet et laissez bien refroidir.\\
\\

\end{minipage}

\begin{minipage}[c]{\textwidth}
\recette{Prunes au vinaigre}
Piquer 4kg de prunes et les équeuter. Ajouter deux litres de vinaigre de vin, 1 kg de sucre Candy blanc, du poivre, deux clous de girofle, du thym et du macis (écorce de muscadier).\\
Le sirop doit bouillir 4 minutes. Ajouter les fruits et faire bouillir. Débarrasser dans une urne.\\
Deux jour après, rangez les prunes en bocaux. Refaire bouillir le jus que vous versez sur les prunes. Bien boucher solidement le lendemain\\
\\

\end{minipage}

\begin{minipage}[c]{\textwidth}
\recette{Morilles}
Les laver à plusieurs reprises à l’eau avec grand soin. Bien éponger. Les sauter au beurre. Comme elles donnent beaucoup d'eau, les faire sauter deux fois puis les égoutter jusqu'à ce qu'elles nagent.\\
Les jeter dans un nouveau beurre brulant pour finir la cuisson.\\
(l'eau sert pour un potage ou une sauce)\\
\\

\end{minipage}

\begin{minipage}[c]{\textwidth}
\recette{Potage de tomate passée}
Ajouter 2 lanières de poivron finement coupé et un peu de riz.\\
Jeter le tout dans la soupe\\
\\

\end{minipage}

\begin{minipage}[c]{\textwidth}
\recette{Cèpes à la bordelaise}
Faire sauter dans de l'huile d'olive bien chaude les têtes de cèpes pendant 10 minutes.\\
Quand elles sont dorées, ajouter de l'ail, du persil, les queues hachées, du sel, du poivre et un jus de citron. Servir bien chaud\\
\\

\end{minipage}

\begin{minipage}[c]{\textwidth}
\recette{Potage tortue}
Faire colorer dans une cuillère a soupe : du beurre, 1 kg d'os concassé, une carotte coupée en petits morceaux, 1 oignon, 1 poireau, 1 petit cèleri et 1 gousse d'ail.\\
Puis versé 2 L d'eau, du thym, du laurier, sel, poivre.\\
Faire cuire à cuisson lente pendant 2h puis ajouter 1 dcl de purée de tomate. Cuire encore 15 min. Passer finement. Couper en dés 100g de tête de veau cuite avec 1 dcl de madere et deux jaunes d'oeufs durs. Couper en quartiers puis chauffer.\\
Lier le potage avec une cuillère a soupe et demie de fécule, délayer dans du madere.\\
Rebouillir 5 min.\\
\\

\end{minipage}

\begin{minipage}[c]{\textwidth}
\recette{Vrai consommé}
Dans un bouillon (cuit 2h30) ajouter une tranche de culotte de bœuf haché de 500g trituré avec deux blancs d'œufs, 1 blanc de poireau haché, une carotte et remettre à cuire pendant 1h30, pas plus !\\
\\

\end{minipage}

\begin{minipage}[c]{\textwidth}
\recette{Potage au champignons}
Dans le bouillon d'oignon verser les champignons hachés et cuits à l'eau légèrement salée avec leur eau de cuisson ou tout simplement l'eau de cuisson seule et employer les champignon pour autre chose, lié avec de la farine.\\
\\

\end{minipage}

\begin{minipage}[c]{\textwidth}
\recette{Cassolettes de saint Flour }
Hacher grossièrement 125g de fromage cantal. Y mélanger dans une terrine bien énergiquement : 250g de fromage blanc égoutté, 2 jaunes d'oeufs et un blanc,1/2 cuillère à café de sel fin,6 à 7 tours de moulin à poivre,1 cuillère à café de ciboulettes haché.\\
Distribuer le tout dans des ramequins. Cuire à feu vif pendant 20 min. Servir en saupoudrant de deux cuillères à café de ciboulette hachées.\\
NB : peut se faire avec du gruyère, du reblochon ou autre fromage cuit\\
\\

\end{minipage}

\begin{minipage}[c]{\textwidth}
\recette{Potage aux lentilles}
Avec des os de mouton, faite un bouillon avec un oignons, deux blancs de poireaux, du thym et du laurier. Cuire 9 pommes de terres. Cuire 250g de lentilles à l'eau froide.\\
Ajouter le tout au reste pour cuire ensemble.\\
\\

\end{minipage}

\begin{minipage}[c]{\textwidth}
\recette{Potage au gruau}
Faire revenir quelques minutes un petit oignon dans une noix de beurre mousseuse.\\
Couvrir d'eau. Jeter y un cèleri finement coupé et une forte pognée de gruaux, du thym, du laurier de l’ail et un cube.\\
Passer et servez\\
\\

\end{minipage}

\begin{minipage}[c]{\textwidth}
\recette{Potage à l'oignion}
Faire revenir dans du beurre mousseau deux oignons hachés jusqu'a ce qu'ils donnent leur suc. Mouiller d'eau bouillante. Laisser cuire 2 minutes. Egoutter. Lier le jus avec 3 cuillères de farine bien délayée et un morceau de liebig.\\
Faire un bon bouillon et servez avec du fromage râpé.\\
\\

\end{minipage}

\begin{minipage}[c]{\textwidth}
\recette{Potage longchamp}
250g de pois réduis en purée\\
100g d’oseilles ? cuites à part au beurre bien fondu. Passer l'oseille et les pois.\\
D’autre part dans deux cubes Liebig, faire cuire du vermissels.\\
Ajouter au potage passé, une noix de beurre et une cuillère de cerfeuil haché, de la\\
crème ou du lait facultatif.\\
\\

\end{minipage}

\begin{minipage}[c]{\textwidth}
\recette{Coulis niçois}
Faire revenir 100g d'oignon dans du beurre mousseux, saupoudrer d’une cuillère de farine. Ensuite, ajouter une boite de purée de tomates ou 5 tomates fraiches et une tranche d'estragon.\\
Verser par dessus 1/2 litre de bouillon ou cube.\\
Finir par un morceau de beurre et servir.\\
\\

\end{minipage}

\begin{minipage}[c]{\textwidth}
\recette{Potage Longchamp}
\ingredients{
\item 250g de pois réduis en purée
\item 100g d’oseille cuite à part au beurre bien fondu
}
Passer l'oseille et les pois\\
D’autre part dans deux cube Liebig faite cuire des vermicelles\\
Ajouter au potage passé + une noix de beurre et une cuillère de cerfeuil haché\\
Crème ou lait facultatif\\
\\

\end{minipage}

\begin{minipage}[c]{\textwidth}
\recette{Potage aux champignons}
Dans le bouillon d'oignions verser les champignons haché est cuit à l'eau légèrement salé\\
Avec leur eau de cuisson ou tout simplement l'eau de cuisson seul et employer les champignons pour autre chose, lié avec de la farine\\
\\

\end{minipage}

\begin{minipage}[c]{\textwidth}
\recette{Potage à l'oignions}
Faire revenir dans du beurre mousseau deux oignions hachée jusqu'à ce qu'ils donnent leur suc\\
Mouiller d'eau bouillante\\
Laisser cuire 2 minutes\\
Égoutter\\
Lier le jus avec 3 cuillères de farine bien délayé\\
Un morceau de Liebig\\
Faite un bon bouillon et servez avec du fromage râpé\\
\\

\end{minipage}

\begin{minipage}[c]{\textwidth}
\recette{Coulis niçois}
Fais revenir 100g d'oignions dans du beurre mousseux\\
Saupoudrez une cuillère de farine\\
Ensuite mettez une boite de purée de tomate ou 5 tomates fraîches et une tranche d'estragon\\
Verser par dessus 1/2 litre de bouillon ou cube\\
Finir par un morceau de beurre et servir\\
\\

\end{minipage}

\begin{minipage}[c]{\textwidth}
\recette{Œufs Mirabeau}
Longuement battre 4 œufs avec un demi verre de lait tiède\\
150g de fromage râpé\\
Une cuillère de beurre fondu et du poivre\\
Mettre dans des ramequins beurrés\\
Cuire au bain mari dans 2 cm d'eau au four 15 à 20 minutes\\
démouler, servir avec de la béchamel ou de la sauce tomate\\
\\

\end{minipage}

\begin{minipage}[c]{\textwidth}
\recette{Œufs mollet cressonnières}
Cuire les œufs 5 min et les rafraichir avant de les pelés\\
Nettoyer une botte de cressons\\
La jeter dans de l'eau bouillante salée\\
Égoutter de suite et rafraichir\\
Exprimé? L’eau \\
Passer au tournis et mélanger à une sauce mayonnaise très relevé\\
\\

\end{minipage}

\begin{minipage}[c]{\textwidth}
\recette{Guacamole (entré mexicaine)}
Ecraser 3 avocats mur\\
Mélanger avec 25g d'oignons rouge hachée,\\
Quelques gouttes de tabasco,\\
Le jus d’ un demie citron\\
1/2 de cuillère à café d'ail en poudre,\\
1/2 cuillère à café de curry en poudre\\
Une pincée de sel\\
\\

\end{minipage}

\begin{minipage}[c]{\textwidth}
\recette{Confiture d'oignions}
Émincé 2 gros oignions ou plus\\
Les faire revenir dans une noix de beurre en les saupoudrant de 2 cuillère à soupe de sucre\\
Laisser cuire 10 min\\
Les mouiller avec 4 cuillère à soupe de vinaigre de vin, un demie litre de vin rouge, 4 cuillère à soupe de grenadine ou 4 cuillère à café de gelé de groseille.\\
Saupoudrez de thym laurier sel poivre\\
Laisser réduire 1/2 h jusqu'à obtenir une masse rouge\\
\\

\end{minipage}

\begin{minipage}[c]{\textwidth}
\recette{Beignet soufflé au parmesan}
Mettre dans une casserole 60g de beurre, du sel, du poivre, de la muscat, du paprika et les 3/4 d'un verre contenant un mélange lait et eau\\
Faire bouillir\\
Ajouter en remuant a la spatule 150g de farine jusqu’à obtenir une pâte épaisse\\
Continuer à battre jusqu’à ce que la pâte épaisse se détache en boule de la casserole\\
Hors du feu ajouter deux œufs à la fois en remuant jusqu'à   ce qu'elle redevienne épaisse. Ajouter en battant 60g de parmesan râpé\\
Cuire à la friture des boulettes de pate grosse comme une noix\\
Servez bien chaud sur serviette pliée ou papier absorbant\\
\\

\end{minipage}

\begin{minipage}[c]{\textwidth}
\recette{Conserve de cornichons et petits oignons}
Les petits oignons doivent être pelés. Les mettre dans une terrine avec une poignée de sel - bien mélanger - laisser reposer une nuit. \\
Le lendemain, bien essuyer 1 ou deux fois selon la saleté. Les mettre en bocal. Ajouter du vinaigre, de l'estragon, du fenouil  éventuellement un piment.\\
\\

\end{minipage}

\begin{minipage}[c]{\textwidth}
\recette{Brochette créole }
\ingredients[(4 personnes)]{
\item 16 petits cubes de gouda d’environ 1 cm 50
\item 16 très fine tranches de lard fumé
\item 4 tranches d'ananas et 4 piques à brochette
}
Enroulez les cubes de fromage entièrement dans le lard\\
Coupé les tranche d'ananas En cube,\\
Enfilé alternativement les cube de fromage enroulé de lard et ceux d'ananas\\
Mettre 10 minutes au grill ou sur un poêle chaude\\
Retourner à mi-cuisson\\
Servir très chaud (avec du riz ça peut faire un repas complet)\\
\\

\end{minipage}

