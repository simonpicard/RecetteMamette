\begin{minipage}[c]{\textwidth}
\recette{Sauce vinaigrette}
\ingredients{
\item Huile. 
\item 1 cuillère à café de moutarde.
\item 1 bonne cuillère à soupe de vinaigre.
}
Mélanger le tout en battant énergiquement pour obtenir plus ou moins la même consistance qu’une mayonnaise.\\
\\

\end{minipage}

\begin{minipage}[c]{\textwidth}
\recette{Sauce béarnaise}
\ingredients{
\item 2 jaunes d’œufs
\item 2 cuillères à soupe de sauce (quelle sauce ?) 
\item 1 pincée de sel. 
}
Mélanger le tout à l’aide d’un fouet. Cuire à feu doux en tournant toujours (ne pas laisser bouillir !). Ajouter du beurre par petits morceaux tout en fouettant.\\
\\

\end{minipage}

\begin{minipage}[c]{\textwidth}
\recette{Sauce blanche pour légumes}
Fondre du beurre.\\
Y ajouter $\nicefrac{1}{2}$ cuillère à soupe de farine et une pincée de sel.\\
Sur feu doux ajouter du lait en tournant. \\
Y Verser les légumes cuits à l’eau.\\
\\
NB : pour les épinards ajouter 1 cuillère à soupe de farine en plus pour allonger le légume. \\
\\

\end{minipage}

\begin{minipage}[c]{\textwidth}
\recette{Sauce mousseline (artichauts)}
Faire fondre au bain-marie 150 gr de beurre.\\
Faire réduire de moitié 4 cuillères à soupe de vinaigre.\\
Laisser tiédir.\\
Ajouter 2 jaunes d’œufs en fouettant.\\
Ajouter peu à peu le beurre fondu en fouettant toujours. \\
Saler.\\
Ajouter un peu de jus de citron.\\
Ajouter le reste du beurre.\\
Garder au chaud au bain-marie.\\
\\

\end{minipage}

\begin{minipage}[c]{\textwidth}
\recette{Sauce hollandaise (asperges) }
Mettre dans un poêlon $\nicefrac{1}{2}$ cuillère à soupe d’eau, 1 forte pincée de sel, le jus d’$\nicefrac{1}{2}$ citron (environ 1 cuillère à soupe) et 1 jaune d’œuf. \\
Mélanger alors seulement au fouet.\\
Mettre à feu doux en mélangeant au fouet jusqu’à début d’épaississement.\\
Incorporer peu à peu 60 gr de beurre en mélangeant bien, à très petit feu. \\
Mélanger sans arrêt.\\
\\
NB : Ne pas verser  la sauce dans une saucière plus chaude que la sauce.\\
\\

\end{minipage}

\begin{minipage}[c]{\textwidth}
\recette{Fines herbes}
Mélanger un peu de cressonnette, du persil, du cerfeuil, quelques feuilles d'épinard et d’oseille, 1 ou 2 petits oignons et 1 ou 2 feuilles d’artichaut (sans la partie blanche). Le tout doit peser environ 25 gr.\\
Hacher fin.  \\
\\
Peut s'employer par exemple en mélangeant les fines herbes à de la mayonnaise et en versant le mélange sur de la tête de veau (environ 150 gr par personnes) coupée en cube.\\
\\

\end{minipage}

\begin{minipage}[c]{\textwidth}
\recette{Sauce tartare}
\ingredients{
\item $\nicefrac{1}{4}$ L de mayonnaise
\item 1 gros oignon haché fin
\item 15 gr de feuilles d’estragon frais au vinaigre hachées finement
\item 1 cuillère à dessert de sauce anglaise
\item 25 gr de cornichons hachés 
\item 1 cuillère à dessert câpres hachées
\item Poivre et sel
\item $\nicefrac{1}{2}$ cuillère à café de vinaigre à l’estragon 
\item 1 cuillère à café de moutarde
}
Mélanger au fouet et servir bien frais ! \\
\\

\end{minipage}

\begin{minipage}[c]{\textwidth}
\recette{Ketchup aux pommes}
Prendre 12 pommes pelées, vidées et découpées en quartier. Les couvrir d'eau et laisser mijoter jusqu'a ce qu'elles deviennent molles, passer le tout, ajouter par litre de pomme :\\
une tasse de sucre\\
une cuillère à thé de moutarde\\
une cuillère à soupe de sel\\
une cuillère à thé de clous de girofles\\
2 cuillère à thé de canelle\\
2 tasses de vinaigre de cidre\\
2 oignions râpés\\
Laisser mijoter pendant une heure. Mettre en bouteille.\\
\\

\end{minipage}

